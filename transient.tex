Chapter \ref{chap:msre} verified the hybrid $S_N$-diffusion method implemented in Moltres through
1-D, 2-D, and 3-D
$k$-eigenvalue simulations of the \gls{MSRE} against the OpenMC Monte Carlo neutron transport code
for molten salt reactor control rod modeling. The 3-D simulations validated the 3-D \gls{MSRE}
model in this work against \gls{MSRE} experimental data and the \gls[MSRE} numerical benchmark
study \cite{fratoni_molten_2020} in the context of the initial criticality and rod worth
measurement experiments. The 3-D simulations also showed that the hybrid method runs with
reasonable computational costs compared to the standard neutron diffusion method and exhibited
efficient strong scaling across compute nodes.

This chapter builds on the foundations of Chapter \ref{chap:msre} to verify and validate the hybrid
method for time-dependent 3-D simulations of reactivity-initiated transients. This chapter includes
two demonstrations using the hybrid method of the \gls{MSRE} rod drop and reactivity insertion
experiments. Beyond time-dependent modeling, the simulations will also verify and validate
compatibility between the existing neutronics/thermal-hydraulics coupling capabilities in Moltres
and the newly-implemented hybrid method.

Section \ref{sec:rod-drop} covers the model setup, numerical results, and discussion of the rod
drop experiment. Section \ref{sec:reactivity-insertion} covers the model setup, numerical results,
and discussion of the reactivity insertion experiment. Lastly, Section \ref{sec:transient-summary}
summarizes the goals and outcomes in this chapter.

\section{Rod Drop Experiment} \label{sec:rod-drop}

\gls{MSRE} researchers conducted a series of rod drop experiments at three different $^{235}$U
loadings \cite{prince_zero-power_1968}. These experimented were conducted from an initially critical
and low power state with an initial neutron count rate of approximately 30,000 from neutron detectors
placed near the reactor. A rapid-action camera photographed the integral count shown on the
detectors about once every second starting from a few seconds before the rod drop to about thirty
seconds after the rod drop. This work aims to replicate the integral count curve following a rod
drop of Rod 1 with the $^{235}$U loading corresponding to "30 capsule additions" of highly-enriched
$^{235}$U capsules.

\subsection{Model Setup}

\subsection{Rod Drop Numerical Results \& Discussion}

\section{Reactivity Insertion Experiment} \label{sec:reactivity-insertion}

\subsection{Model Setup}

\subsection{Reactivity Insertion Numerical Results \& Discussion}

\section{Summary} \label{sec:transient-summary}

Chapter \ref{chap:msre} verified the hybrid $S_N$-diffusion method implemented in Moltres through
1-D, 2-D, and 3-D
$k$-eigenvalue simulations of the \gls{MSRE} against the OpenMC Monte Carlo neutron transport code
for molten salt reactor control rod modeling. The 3-D simulations validated the 3-D \gls{MSRE}
model in this work against \gls{MSRE} experimental data and the \gls{MSRE} numerical benchmark
study \cite{fratoni_molten_2020} in the context of the initial criticality and rod worth
measurement experiments. The 3-D simulations also showed that the hybrid method runs with
reasonable computational costs compared to the standard neutron diffusion method and exhibited
efficient strong scaling across compute nodes.

This chapter builds on the foundations of Chapter \ref{chap:msre} to validate the hybrid
method for neutronics modeling in time-dependent 3-D simulations of reactivity-initiated
transients. This chapter presents simulation results from Moltres for the \gls{MSRE} rod drop
experiment. Beyond time-dependent modeling with the hybrid method, the simulations will also
demonstrate compatibility between the existing neutronics/thermal-hydraulics coupling
capabilities in Moltres and the newly-implemented hybrid method.

Section \ref{sec:rod-drop} covers the model setup, numerical results, and discussion of the rod
drop experiment.
Section \ref{sec:transient-summary}
summarizes the findings from the rod drop experiment study.

\section{Rod Drop Experiment} \label{sec:rod-drop}

\gls{MSRE} researchers conducted a series of rod drop experiments at three different $^{235}$U
loadings \cite{prince_zero-power_1968}. These experiments were conducted from an initially critical
and low power state with an initial neutron count rate of approximately 30,000 counts per second
from neutron detectors placed near the reactor. A rapid-action camera photographed the integral
count shown on the detectors about once every second starting from a few seconds before the rod
drop to about thirty seconds after the rod drop.

This work aims to replicate the integral count curve following a rod drop of Rod 1 with the
$^{235}$U loading corresponding to ``30 capsule additions'' of highly-enriched $^{235}$U capsules
into the \gls{MSRE} through the ``pump bowl''. Each capsule contains 85 g of $^{235}$U, of which
94.94 \% enters the core
due to residual fuel salt remaining in the drain tanks after mixing \cite{fratoni_molten_2020}.

\subsection{Model Setup} \label{sec:rod-drop-setup}

The key experimental parameters to simulate the rod drop experiment are: the exact $^{235}$U
loading, the initial and final rod heights, the total reactivity withdrawn, and the speed of the
rod drop.

The \gls{MSRE} achieved initial criticality after eight capsule additions with Rod 1 inserted to a
height of 46.6 inches relative to the fully inserted height \cite{prince_zero-power_1968}.
From the \gls{MSRE} experimental rod worth
data, the rod worth evaluates to approximately 72 pcm relative to the fully withdrawn state; the
$k_\text{eff}$ of the \gls{MSRE} at initial criticality with Rod 1 fully withdrawn would be
1.00072. According to the same \gls{MSRE} report, the
authors back-calculated the critical $^{235}$U loading with Rod 1 fully withdrawn to be 65.25 kg.
They also provided a $^{235}$U concentration reactivity coefficient of 0.223 $(\delta k/k)/(\delta m
/m)$, where $k$ is the multiplication factor and $m$ is the $^{235}$U mass. The excess $^{235}$U
mass at initial criticality compensating for Rod 1 inserted to 46.6
inches can estimated through the following steps:
%
\begin{align}
  \frac{\Delta k / k}{\Delta m / m} &= 0.223 \nonumber \\
  \frac{m}{k}\frac{dk}{dm} &= 0.223 \nonumber \\
  \int^{1.00072}_1 \frac{1}{k}\ dk &= 0.223 \int^{65.25+\Delta m}_{65.25} \frac{1}{m}\ dm \nonumber \\
  \ln \left(1.00072\right) &= 0.223 \ln\left(\frac{65.25+\Delta m}{65.25}\right) \nonumber \\
  \Delta m &= 0.2109 \mbox{ kg}.
\end{align}
%
This change in mass is equivalent to approximately 2.61 highly-enriched $^{235}$U capsules.
Therefore, the $^{235}$U loading for the rod drop experiment is:
%
\begin{gather}
  m_\text{rod drop} = 65.25 \text{kg} + (30 - 8 + 2.61) * 0.085 \text{kg} * 0.9494 = 67.24 \text{kg}
\end{gather}
%
The $k_\text{eff}$ of the \gls{MSRE} with 67.24 kg loading and all rods fully withdrawn is:
%
\begin{align}
  \int^{k_{67.24 \text{kg}}}_1 \frac{1}{k}\ dk &= 0.223 \int^{67.24}_{65.25} \frac{1}{m}\ dm \nonumber \\
    k_{67.24 \text{kg}} &= 1.00671
\end{align}

\begin{figure}[h]
  \centering
  \includegraphics[width=0.6\columnwidth]{msre-rod-worth}
  \caption{Integral control rod worth of Rod 1 \cite{prince_zero-power_1968} at critical and final
  $^{235}$U loading. The final loading rod worth curve is scaled down by 1.087 relative to the
  critical loading rod worth curve.}
  \label{fig:msre-rod-worth-2}
\end{figure}

When loading more $^{235}$U into the core, the shape of the integral rod worth does not change,
but the magnitude scales down by a factor of 1.087 between the critical and final loading
as shown in Figure \ref{fig:msre-rod-worth-2}. After applying this scaling factor to the monotone cubic
spline interpolated \gls{MSRE} rod worth curve, the initial height of Rod 1 to keep the \gls{MSRE}
with 67.24 kg loading critical is approximately 30.75 inches. During the rod drop experiment, the rod
drops from this height to its fully inserted state at 0 inches. This drop induces a total
reactivity withdrawal of 1500 pcm.

The rod drop simulation uses the hybrid $S_N$-diffusion method in Moltres on the verified 3-D
\gls{MSRE} model from Chapter
\ref{chap:msre} at the critical $^{235}$U loading. Refer to Section \ref{sec:3d-model-setup} for
details on the 3-D model. The initial rod height for the \gls{MSRE} model to induce a similar total
reactivity withdrawal of 1500 pcm from Rod 1 drop is 29.4 inches. The acceleration of the
rod is approximately 15 ft s$^{-2}$ from prior testing \cite{prince_zero-power_1968}. At this
acceleration, the rod reaches its fully inserted height at $t=0.4046$ s. Figure \ref{fig:rod-height}
shows the evolution of rod height at each timestep within the first second of the rod drop simulation.

\begin{figure}[htb!]
  \centering
  \includegraphics[width=0.7\columnwidth]{rod-height}
  \caption{Evolution of rod height evaluated at each timestep within the first second of the rod
  drop simulation. The rod reaches its fully inserted height at $t=0.4046$ s.}
  \label{fig:rod-height}
\end{figure}

A significant change for the rod drop simulation compared to the $k$-eigenvalue 3-D simulation,
other than the time-dependence, is the inclusion of \glspl{DNP} for the delayed neutron response.
I used the \gls{MOOSE} MultiApp capability available in Moltres to set up a nested iterative
coupling structure for the hybrid
$S_N$-diffusion method with \glspl{DNP} such that the neutron diffusion and \gls{DNP} solvers are
coupled through inner iterations and the $S_N$ solver is coupled them through outer iterations.
Figure \ref{fig:rod-drop-coupling} illustrates this iterative coupling structure and the data
transfers between the solvers. The \gls{DNP} solver is a preexisting capability in Moltres
described in Section \ref{sec:moltres-physics}. For the rod drop simulation, the salt velocity and
\gls{DNP} diffusion coefficients were set to zero, i.e., static salt condition. The simulation
grouped the \glspl{DNP} into six \gls{DNP} decay groups automatically determined by the
ENDF/B-VII.1 nuclear data library \cite{chadwick_endf/b-vii.1_2011} and OpenMC during group
constant generation.

\begin{figure}[t]
  \tikzstyle{every node}=[font=\small]
  \centering
  \begin{tikzpicture}
    \node (1) [solver] {\textbf{$\bm{S_N}$ transport subsolver}};
    \node (2) [solver, right of=1, xshift=4cm] {\textbf{Neutron\\ diffusion\\ subsolver}};
    \node (3) [solver, right of=2, xshift=4cm] {\textbf{DNP subsolver}};
    \draw [arrow] (1.10) -- node[anchor=south, text width=4cm, align=center] {$\vec{D}_g, \gamma_g$} (2.170);
    \draw [arrow] (2.190) -- node[anchor=north, text width=4cm, align=center] {$\phi_g, C_i$} (1.350);
    \draw [arrow] (2.10) -- node[anchor=south, text width=4cm, align=center] {$Q_f, Q$} (3.170);
    \draw [arrow] (3.190) -- node[anchor=north, text width=4cm, align=center] {$C_i$} (2.350);
    \node (5) [bound, fit=(2) (3), inner sep=8pt, label={[align=left]above:\textbf{Inner iteration}}] {};
    \node (6) [bound, fit=(1) (5), inner sep=15pt, label={[align=left]above:\textbf{Outer iteration}}] {};
  \end{tikzpicture}
  \caption{The nested iteration structure coupling the $S_N$, neutron diffusion, and \gls{DNP}
  solvers for the rod drop simulation using the hybrid $S_N$-diffusion method.}
  \label{fig:rod-drop-coupling}
\end{figure}

The output of a $k$-eigenvalue simulation with the rod at the prescribed initial height provides
the initial group flux and \gls{DNP} distributions for the rod drop simulation. The time-depedent
rod drop simulation uses the \gls{MOOSE} Restart capability to load the variable distributions from
the prior $k$-eigenvalue simulation. The total fission neutron production rate at every timestep
is normalized to match the 30,000 counts per second at $t=0$ s from the \gls{MSRE} rod drop
experimental measurements.

\subsection{Rod Cusping Effect Correction}

In deterministic codes with spatial discretization, the material interfaces of a partially-inserted
control rod may not always align with mesh elements. This results in a heterogenous
mesh element that is composed of a control rod material at the top and a non-rod material at the
bottom (for control rods that are inserted from above). The rod cusping effect occurs when
homogenizing mixed mesh elements using the volume weighting method to combine dissimilar material
neutron group constants. Rod cusping error typically decreases with increasing axial mesh
resolution because the size of the mixed mesh elements become smaller relative to the entire
problem domain.

Several rod cusping correction methods exist in literature that are variations of the flux-volume
weighting approach to generate more accurate homogenized group constants in the mixed mesh
elements \cite{yamamoto_cell_2004, graham_subplane_2018, schunert_control_2019}. However, due to
a failed flux-volume weighting implementation and time constraints, this work uses an empirical
approach for rod cusping effect correction. The empirical approach involves precomputing the
$k_\text{eff}$ with the control rod at various levels of insertion within a mesh element. These
simulations apply a simple volume weighting approach to compute a homogenized group constant value
as follows:

\begin{figure}[t]
    \centering
    \begin{subfigure}[t]{.49\textwidth}
        \centering
        \includegraphics[width=\textwidth]{rod-cusping-1}
        \caption{Uncorrected volume weighting}
        \label{fig:rod-cusping-1}
    \end{subfigure}
    \hfill
    \begin{subfigure}[t]{.49\textwidth}
        \centering
        \includegraphics[width=\textwidth]{rod-cusping-2}
        \caption{Corrected volume weighting}
        \label{fig:rod-cusping-2}
    \end{subfigure}
    \caption{Fraction of reactivity change against control rod volume fraction in mixed mesh
    elements with the rod inserted at mid-reactor height (27.4 to 29.4 inches) and full insertion
    height (0.0 to 1.9 inches).}
    \label{fig:rod-cusping}
\end{figure}

\begin{gather}
  \Sigma_\text{vw} = \frac{V_\text{r} \Sigma_\text{r} + V_\text{nr} \Sigma_\text{nr}}{V_\text{r} + V_\text{nr}} \label{eq:vw}
  \shortintertext{where}
  \begin{align*}
    \Sigma_\text{vw} &= \mbox{ volume-weighted group constant in mixed mesh element,} \\
    V_\text{r} &= \mbox{ rod volume fraction in mixed mesh element,} \\
    \Sigma_\text{r} &= \mbox{ rod group constant value,} \\
    V_\text{nr} &= \mbox{ non-rod volume fraction in mixed mesh element,} \\
    \Sigma_\text{nr} &= \mbox{ non-rod group constant value.}
  \end{align*}
\end{gather}

Note that vertical material and mesh interfaces in the 3-D \gls{MSRE} model always align perfectly
because the control rod moves vertically. Therefore, the mixed mesh elements occur only around the
bottom tip of the control rod. Additionally, the volume fraction is equivalent to the height
fraction, and all mixed mesh elements at the same height share the same volume fraction. Figure
\ref{fig:rod-cusping-1} shows how the fraction of reactivity change changes with the control rod
volume fraction when the control rod is at around mid-reactor and full insertion heights. The data
points depict the nonlinear relation between the reactivity change and volume fraction that the
``cusping effect'' is named after. Within this small range of rod motion, the relation should be
approximately linear.

Recognizing that the rod cusping effect approximately follows a $y=x^p$ curve profile, I applied
curve-fitting to obtain the reactivity changes using corrected volume fractions shown in Figure
\ref{fig:rod-cusping-2}. The corrected volume fraction $V_r'$ is calculated as follows
%
\begin{align}
  V_r' &= cV_r^p
  \shortintertext{where}
  p &= 2.5148, \nonumber \\
  c &= 0.43807. \nonumber
  \end{align}
%
The corrected volume fraction produces the desired approximately linear relation between reactivity
change and control rod volume fraction. While this empirical approach worked reasonably well for
this study, a more robust technique will be necessary for future control rod simulations beyond
this work.

\subsection{Rod Drop Numerical Results \& Discussion}

Figure \ref{fig:count-rate} shows the neutron count rate following the rod drop from Moltres. It
shows the steep initial decline in neutron count rate as the rod drops further into the core. The
decline slows at around $t=0.5$ s due to the delayed response of \glspl{DNP}. During this
time-dependent simulation, convergence issues prevented it from converging beyond $t=0.825$ s. The
simulation managed to continue after raising the convergence tolerance value of the neutron
diffusion subsolver and the fixed point iterations from $\epsilon_\text{tol}=10^{-8}$ to $10^{-6}$.
However, this action led to under-converged scalar flux and neutron count results between $t=1.325$
s and $3.825$ s.

Figure \ref{fig:integral-count} compares the integral neutron count during the rod drop experiment
from \gls{MSRE} experimental data and the hybrid method numerical results. The hybrid method
reproduces the expected shape of the integral
neutron count characterized by the drop in neutron count rate leading to a gradual flattening of
the integral count curve. Moltres underpredicts the integral neutron count throughout the
simulation compared to the \gls{MSRE} experimental data.
Experimental uncertainties surrounding the rod drop experiment include the actual $^{235}$U
concentration, initial \& final rod heights, rod acceleration, and the initial neutron count rate.
These uncertainties may have affected the reactivity withdrawal estimates in Section
\ref{sec:rod-drop-setup}.

Considering that the \gls{DNP} subsolver convergence remained unaffected throughout the simulation,
it would have produced \gls{DNP} distributions that are fairly accurate. Only minor discrepancies
may be present in the \gls{DNP} distributions through \gls{DNP} production from under-converged
scalar fluxes. Figure \ref{fig:count-rate-interp} shows the neutron count rate if we replaced the
under-converged data points between $t=1.325$ s and $3.825$ s with linearly interpolated count
rates from $t=0.825$ s to $4.325$ s. Figure \ref{fig:integral-count-interp} shows the corresponding
integral neutron count curve generated using the interpolated data. In this case, the integral
count curve shape is more consistent with the experimental data, albeit seemingly in response to a
larger reactivity withdrawal. Sources of experimental uncertainties listed in the previous
paragraph may have contributed to the discrepancy between the experimental and numerical data.

\begin{figure}[p]
  \centering
  \includegraphics[width=0.78\columnwidth]{count-rate}
  \caption{Neutron count rate during the rod drop experiment from Moltres. The convergence tolerance
  of the neutron diffusion subsolver and the fixed point iterations was raised from $\epsilon=10^{-8}$
  to $10^{-7}$ from $t=1.32$ s onwards.}
  \label{fig:count-rate}
  \includegraphics[width=0.78\columnwidth]{integral-count}
  \caption{Integral neutron count during the rod drop experiment from \gls{MSRE} experimental data
  and Moltres.}
  \label{fig:integral-count}
\end{figure}

\begin{figure}[p]
  \centering
  \includegraphics[width=0.78\columnwidth]{count-rate-interp}
  \caption{Neutron count rate during the rod drop experiment from Moltres with linearly interpolated
    data between $t=1.325$ s and $3.825$ s.}
  \label{fig:count-rate-interp}
  \includegraphics[width=0.78\columnwidth]{integral-count-interp}
  \caption{Integral neutron count during the rod drop experiment from \gls{MSRE} experimental data
  and Moltres with linearly interpolated count rate data between $t=1.325$ s and $3.825$ s.}
  \label{fig:integral-count-interp}
\end{figure}

Lagging solution states during the first fixed point iteration in each time-step also slowed
solution convergence, particularly when the control rod was in motion. This limited the simulation
from taking larger time-steps. Further work could include rearranging the nested coupling structure
or implementing relaxation schemes to improve solver numerical stability and convergence.

\FloatBarrier

\section{Summary} \label{sec:transient-summary}

The hybrid $S_N$-diffusion method combines the strengths of the $S_N$ and neutron
diffusion methods to accurately model control rods while remaining tractable on modest computational
resources. This chapter covers work demonstrating the hybrid $S_N$-diffusion method through a
time-dependent reactivity-initiated simulation. The hybrid $S_N$-diffusion solver implemented in
Moltres is coupled to preexisting \gls{DNP} and temperature solver capabilities to model
transient behavior in reactors.

The time-dependent simulation is based on a zero-power \gls{MSRE} rod drop experiment. The
rod drop is estimated to induce a 1500 pcm reactivity withdrawal in the \gls{MSRE}. Preliminary
calculations determined the required initial rod height in the numerical model to replicate the
same reactivity withdrawal effect and eliminate rod worth discrepancies. Following an initial
$k$-eigenvalue simulation to obtain initial flux and \gls{DNP} distributions, the rod drop
simulation ran with total fission neutron production rate measurements at each timestep. The
integral neutron production, normalized to match the initial experimental neutron count rate,
showed good agreement with the experimental integral neutron count curve following the rod drop.
Convergence issues with the nested coupling structure of the hybrid method and the \gls{DNP}
subsolver impacted the precision of several data points during the simulation.
Future investigations could look into alternative nested solver coupling structures and relaxation
schemes to resolve these issues.
Experimental uncertainties in the actual $^{235}$U concentration, initial \& final rod heights, rod
acceleration, and the initial neutron count rate were also responsible for discrepancies relative
to experimental data.
Overall, this work has successfully demonstrated the hybrid method for coupled time-dependent
reactor analysis of reactivity-initiated transients involving control rods.

Five decades since the \gls{MSRE} \cite{haubenreich_experience_1970}, \glspl{MSR} are once again
drawing significant interest as an advanced reactor concept with enhanced safety, sustainability,
efficiency, and cost aspects over the
current generation of predominantly \glspl{LWR}. This observation is supported by the
nuclear start-ups developing \gls{MSR} designs and public funding from various governments around
the world in the race to reduce carbon emissions. \gls{MSR} simulation tools are essential in
supporting this wave of \gls{MSR} research and development and reaching the eventual \gls{MSR}
commercialization to meet our growing energy needs.
\glspl{MSR} feature physics interactions, such as \gls{DNP} flow, turbulent heat transport in fuel
salt regions, and strong temperature reactivity feedback, that may render some conventional reactor
software less suitable for \gls{MSR} analysis.

The overarching goal of this work was to enhance Moltres, a \gls{MOOSE}-based application, as a
reliable, intermediate-fidelity simulation tool for multiphysics \gls{MSR} analysis that scales
well from workstations to larger computing clusters. I successfully met this goal through two main
objectives:
%
\begin{enumerate}
  \item Verification and validation of multiphysics capabilities in Moltres
  \item Development of a novel hybrid $S_N$-diffusion method for improved control rod modeling in
    Moltres
\end{enumerate}

I conducted two \gls{VV} studies on Moltres to meet the first objective.
The first study follows the CNRS benchmark
study proposed and investigated by several European research groups \cite{tiberga_results_2020}.
The second study is a collaborative effort with Dr Aaron Reynolds (formerly of Oregon State
University) based on the \gls{MSRE} zero-power pump experiments \cite{prince_zero-power_1968}.
I also implemented and verified a Spalart-Allmaras turbulence model in Moltres to enable future
turbulent salt flow simulations. Chapter \ref{chap:verification} covers the \gls{VV} studies and
turbulence model implementation. Chapters \ref{chap:hybrid}, \ref{chap:msre}, and
\ref{chap:transient} cover the development, verification, and demonstration of the hybrid
$S_N$-diffusion method to meet the second objective.

Chapter \ref{chap:verification} provided a detailed description of Moltres and its existing
capabilities prior to this work. Moltres is a multiphysics simulation software for \glspl{MSR}. It
has also been adopted, by itself or coupled to other \gls{MOOSE} applications, for high-temperature
gas-cooled reactor and lead-cooled fast reactor analyses
\cite{fairhurst-agosta_multi-physics_2020, ji_numerical_2024}. Moltres features a modular code
architecture that enables the integration of its own multigroup neutron diffusion and \gls{DNP}
models with the Squirrel \gls{MOOSE}-based application \cite{noauthor_arfcsquirrel_nodate} and the
\gls{MOOSE} Navier-Stokes module \cite{peterson_overview_2018} for thermal-hydraulics modeling.
The \gls{MOOSE} framework provides user- and developer-friendly interfaces that enable fast
prototyping and implementation of experimental modeling features.
The underlying libMesh \cite{kirk_libmesh_2006} and PETSc \cite{satish_petsc_2019} libraries
provide state-of-the-art meshing and \gls{FEM} solver capabilities that are robust, efficient, and
highly scalable.

Chapter \ref{chap:verification} continued with the CNRS benchmark verification results from
Moltres. The CNRS benchmark \cite{tiberga_results_2020} is a numerical benchmark study
designed to assess multiphysics coupling in \gls{MSR} simulation tools for fast-spectrum
\gls{MSR} modeling. The benchmark starts with three steady-state single-physics problems, followed by four
steady-state multiphysics problems incrementally introducing various multiphysics coupling
interactions, and lastly a series of transient multiphysics problems for dynamic analysis. Moltres
showed good agreement with the four participating \gls{MSR} simulation tools published in the CNRS
benchmark publication. Most of the test metric measurements from Moltres fell within one standard
deviation of the benchmark average. The only discrepancies arose in temperature readings along the
top boundary of the 2-D model due to discontinuous and unphysical velocity boundary conditions
prescribed by the leaky lid-driven cavity flow profile. All spatial distributions of fission rate
density, delayed neutron source, temperature, and velocity were nearly indistinguishable compared
to the benchmark participants.

Chapter \ref{chap:verification} also covered the collaboration study with the developer of
QuasiMolto \cite{reynolds_analysis_2023} for a \gls{VV} study based on the \gls{MSRE} zero-power
pump start-up and coast-down experiments \cite{prince_zero-power_1968}. This study was designed to
assess the looped \gls{DNP} flow modeling capabilities in Moltres and QuasiMolto. QuasiMolto is a
circulating fuel reactor simulation software that employs a multi-level quasi-diffusion method
\cite{tamang_multilevel_2014} for 2-D axisymmetric multiphysics simulations of \glspl{MSR}.
We developed a 2-D axisymmetric model of the \gls{MSRE} for this study. Despite the widely
different numerical schemes employed by Moltres and QuasiMolto, both codes were highly consistent
with each other in $k$-eigenvalue neutronics simulations under static and steady salt flow
conditions prior to pump transient simulations. The $k_\text{eff}$ estimates from both codes fell
within 22 and 21 pcm of each other under static and steady flow conditions. Consequently, reactivity
loss estimates due to \gls{DNP} decay outside the core agreed within 1 pcm. Spatial flux and
\gls{DNP} distributions showed nearly perfect overlap with relative differences under 0.8 \%. Both
codes remained consistent with each other in the pump start-up and coast-down transient simulations
when comparing the reactivity loss over time induced by changes in \gls{DNP} flow. The numerical
results deviated from \gls{MSRE} experimental data due to the absence of upper and lower plena in
the numerical models and the omission of turbulent mixing. These affected the peak reactivity loss
measurements and the flattening of reactivity oscillations in the pump start-up experiment.
Nevertheless, the distinct reactivity oscillations serve as valuable reference points for
verifying \gls{DNP} advection modeling accuracy between Moltres and QuasiMolto, and for other
researchers looking to verify their \gls{MSR} simulation tools.

Chapter \ref{chap:verification} concluded with the implementation and verification of the
Spalart-Allmaras turbulence model in Moltres for enabling wall-bounded turbulent molten salt flow
modeling. The implementation includes a rotation correction scheme
\cite{aupoix_extensions_2003, dacles-mariani_numericalexperimental_1995} for improved accuracy
when modeling turbulent flow with curved streamlines. The model performed well under the turbulent
channel flow verification test \cite{moser_direct_1999}, the turbulent pipe flow validation test
\cite{laufer_structure_1954}, and the backward-facing step flow validation test
\cite{driver_features_1985}.

Chapter \ref{chap:hybrid} introduced the hybrid $S_N$-diffusion method for \gls{MSR} control rod
modeling in time-dependent reactor analyses. The hybrid $S_N$-diffusion method improves on the
standard neutron diffusion method by iteratively applying transport corrections generated from
solving the $S_N$ neutron transport method in subdomains containing highly neutron-absorbing
control rods.
The chapter starts with the derivation for the newly-implemented $S_N$ neutron transport method in
Moltres. It uses the \gls{SAAF} formulation of the $S_N$ equations, which are highly efficient and
scalable with HYPRE multigrid preconditioners \cite{hypre_hypre_2022} from PETSc. I formulated
two possible transport correction schemes for the hybrid method: a diffusion correction and a
drift correction scheme. These schemes fit into modified forms of the neutron diffusion equations
to correct flux errors in near control rods where diffusion theory is not valid. I also developed
an adaptive boundary coupling algorithm to couple the $S_N$ and neutron diffusion solvers. The
algorithm automatically truncates transport correction parameters near the boundaries of the
$S_N$ problem subdomain to discard inaccurate correction parameters and preserve smooth neutron
flux gradients across the interface. The chapter concludes with the numerical implemetation details
of the $S_N$ method with level-symmetric quadrature sets and modifications to the existing neutron
diffusion solver for the hybrid method. The $S_N$ and diffusion solvers are coupled through fixed
point iterations using the \gls{MOOSE} \texttt{MultiApp} and \texttt{Transfers} systems.

Chapter \ref{chap:msre} demonstrated the hybrid $S_N$-diffusion method in $k$-eigenvalue simulations
of 1-D, 2-D, and 3-D \gls{MSRE} models. The 1-D simulations involve six test cases with increasing
geometric complexity modeled after the air-filled control rod thimbles, salt-graphite lattice
structure, and vessel regions in the \gls{MSRE}. Simulations with the OpenMC Monte Carlo neutron
transport code and the $S_8$ method in Moltres provided reference solutions for the hybrid and
neutron diffusion methods to be assessed against. Some discrepancies arose from the eight neutron
energy group structure as evidenced by differences in $k_\text{eff}$ and flux distributions between
OpenMC simulations on continuous energy (OpenMC-CE) and multigroup (OpenMC-MG) modes. Otherwise, the
multigroup $S_8$ method showed good agreement with OpenMC-MG. While $k_\text{eff}$ estimates from
the hybrid methods with the diffusion and drift correction schemes deviated from the $S_8$ method
and OpenMC-MG by 100-400 pcm, the hybrid methods
accurately reproduced control rod worth estimates within 0.5 \% of them (2-3 \% relative to
OpenMC-CE). The neutron diffusion method fared significantly worse at 5.5 \% relative to $S_8$ and
OpenMC-MG, and 8 \% relative to OpenMC-CE. An analysis of the drift and diffusion correction
schemes showed the drift correction scheme to be superior due to singularities and discontinuities
in the diffusion correction parameter. Therefore, the drift correction scheme was chosen for
the remainder of this work. Another analysis showed little impact on control rod worth estimates
from varying the $S_N$-resolved subdomain size as long as the size is kept constant between
$k$-eigenvalue simulations used to calculate the rod worth. The hybrid $S_N$-diffusion method
exhibited a superlinear convergence rate with the number of fixed point iterations, leading to the
$k_\text{eff}$ error estimate falling below $10^{-7}$ after two iterations.

Chapter \ref{chap:msre} continued with 2-D quarter-core and full-core \gls{MSRE} simulations. The
\gls{MSRE} models were nearly identical to the horizontal cross section of the actual \gls{MSRE}
reactor. The hybrid method maintained its accurate control rod modeling capability in the 2-D
simulations involving the incrementally insertion of three control rods in the \gls{MSRE} model.
The hybrid method reported rod worth error magnitudes of less than 40 pcm relative to OpenMC-CE,
which were significant improvements over the neutron diffusion method error estimates ranging from
569 pcm to 1484 pcm. The hybrid method also showed significant improvements in the absolute mean
and maximum errors in fuel channel power distributions over the neutron diffusion method.

The 3-D simulations in Chapter \ref{chap:msre} doubled as validation for the 3-D \gls{MSRE} model
against reference \gls{MSRE} experimental data and the \gls{MSRE} numerical benchmark study
\cite{fratoni_molten_2020} in
the International Reactor Physics Experiment Evaluation Project (IRPhEP) handbook. OpenMC and
hybrid method estimates of $k_\text{eff}$ of the \gls{MSRE} when the experiment first achieved criticality
$k_\text{eff}$ estimates of the \gls{MSRE} at its initial critcality configuration from OpenMC, the
hybrid method, and the neutron diffusion method in this work showed good agreement with the Serpent
model from the \gls{MSRE} numerical benchmark. All numerical estimates exceeded the experimental
value by 1-2 \% due to possible biases and uncertainties in the nuclear data library for graphite
\cite{fratoni_molten_2020}. Temperature reactivity coefficient values from this work also showed
good agreement with \gls{MSRE} data, within experimental uncertainty, with percentage discrepancies
of about 3 \%. In the subsequent control rod worth study, OpenMC and the hybrid method showed nearly
perfect overlap in the integral rod worth curve throughout the entire length of rod travel. Due to
geometric approximations of the control rods in the numerical models, they overestimated the total
rod worth relative to \gls{MSRE} data by approximately 4-5 \%. The hybrid method significantly
outperforms the neutron diffusion method which overestimated the total worth by 21.1 \%.
When comparing solution times, the hybrid method took approximately four times as long as the
neutron diffusion method.
The 3-D hybrid method simulations exhibited nearly linear scaling in strong scaling tests, which
is promising for future projects involving larger reactor models. Some performance optimizations
may be possible to reduce data transfer times between the $S_N$ and diffusion solvers which took
up about 18 \% of the total solution time due to inter-processor communications.

Lastly, Chapter \ref{chap:transient} demonstrated the hybrid method in a time-dependent simulation
based on a \gls{MSRE} zero-power rod drop experiment \cite{prince_zero-power_1968}.
The simulation required coupling the hybrid method solver to the pre-existing \gls{DNP} solver
in Moltres through a nested iteration coupling setup.
I set up the rod drop simulation through a
$k$-eigenvalue simulation with the control rod at its initial height before using the simulation
output as the initial condition for the time-dependent rod drop simulation. Moltres reproduced the
expected prompt and delayed response in the integral neutron count data observed in \gls{MSRE}
experimental data. Convergence issues midway through the simulation had a minor impact on the
solution precision. Overall, this work successfully demonstrated a time-dependent
reactivity-initiated simulation using the hybrid method.

\section{Limitations and Future Work}

The second \gls{VV} study verified the looped \gls{DNP} flow modeling capability in Moltres under
steady-state and transient scenarios. In the original pump experiments, the control rod was driven
by a ``flux servo controller'' that adjusts the rod position in response to flux changes to maintain
criticality. Our study in this work approximated this action through $k$-eigenvalue solvers
coupled to time-dependent \gls{DNP} flow solvers. Potential future work would be to implement a
control system such as a PID (Proportional-Integral-Derivative) controller in combination with
the newly-implemented hybrid method to create a more representative model of the pump experiments.
This would enhance model validation and open up additional research directions which require a
control system.

This work implemented and verified the Spalart-Allmaras turbulence model in Moltres. The turbulence
model \gls{VV} tests indicated significant mesh refinement requirements near the wall. Mesh
refinement scales with the flow Reynolds number. With \gls{MSR} designs such as the \gls{MSFR}
reaching Reynolds numbers on the order of $10^6$, future work in this area should focus on
implementing wall functions. Wall functions eliminate the need for fine mesh near the wall by
approximating the log-law velocity profile near the wall. Verification and demonstration of the
turbulence model, beyond general \gls{VV} tests, in \gls{MSR} turbulent salt flow problems is also
crucial for the continued development of Moltres as a \gls{MSR} simulation tool.

Chapters \ref{chap:msre} and \ref{chap:transient} demonstrated the hybrid $S_N$-diffusion method's
accuracy in control rod worth calculations. However, some discrepancies still persist in the
$k_\text{eff}$ estimates of individual reactor states. These discrepancies affect the average
temperature of the reactor in multiphysics simulations to compensate for the mismatch in
reactivity. These discrepancies could be reduced through
neutron leakage correction at the vacuum boundaries. Remedies for the $k_\text{eff}$ mismatch
exist in a number of forms such as group-wise or matrix albedo boundary conditions.

The 3-D simulations in this work uncovered significant memory usage by the $S_N$ solver, even
with the distributed mesh feature to spread mesh and variable data storage across the compute nodes.
Each full-core simulation required at least 40 nodes with 512 GB memory each to run.
This issue may be a stumbling block for using the hybrid method on smaller computing clusters that
have less memory per processor. A custom preconditioner and solver routine in Moltres could help to
reduce memory usage by referencing the same stored Jacobian C++ variable for neutron angular flux
variables in the same neutron energy group and on the same \gls{FEM} quadrature point. Their
Jacobian formulations are identical aside from the level-symmetric ordinate and weight. Another
area for optimization in the hybrid method is reducing data transfer times between the $S_N$ and
diffusion solvers. Due to mesh distribution, each 3-D simulation spent 18 \% of its solution time
on data transfers between processors. This could be minimized through optimizations in the mesh
distribution and the data transfer caching systems.

Time-dependent simulations in this work uncovered two issues: rod cusping effects when a moving
control rod does not align with the mesh interfaces, and slow solution convergence rates during
control rod motion. While this work applied an empirical technique to correct for rod cusping
effects, Moltres would benefit from a more robust technique for the long term. Slow solution
convergence rates could be resolved by investigating better nested solver coupling structures to
mitigate the effects of lagged solutions and applying solution relaxation schemes.

Finally, future work could demonstrate the hybrid method to its fullest potential through the
simulation of asymmetric reactivity-initiated transients. Asymmetry refers to significant changes
in the neutron flux shape during a transient scenario. The core benefit of the hybrid method is
the spatial resolution it provides to time-dependent simulations involving control rod movement.
This objective could be met through modeling most other \gls{MSR} designs whose control and shim
rods are not centrally located.

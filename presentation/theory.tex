\begin{frame}
  \frametitle{Hybrid $S_N$-Diffusion Method}
  \begin{block}{\textbf{Method Overview}}
    \begin{itemize}
      \item Applies the discrete ordinates ($S_N$) method to a small subregion around a
        control rod to generate corrections for the neutron diffusion equation
      \item Limits computationally expensive $S_N$ calculations to small subdomains
      \item Retains the computational efficiency of the neutron diffusion method
      \item Applies an adaptive algorithm to smooth flux gradients near the $S_N$-diffusion interface
    \end{itemize}
  \end{block}
  \begin{figure}
    \centering
    \includegraphics[width=.7\textwidth]{hybrid-illustration}
    \caption{Illustration of the problem domains of the $S_N$ and neutron diffusion methods in an
    example 1-D problem.}
  \end{figure}
\end{frame}

\begin{frame}
  \frametitle{Hybrid $S_N$-Diffusion Method}
  \textbf{Multigroup Discrete Ordinates $\bm{S_N}$ Neutron Transport Equations}
  \vspace{.2cm}

  The multigroup $S_N$ equations defined on the 3-D spatial domain 
  $\mathcal{D}$ and 2-D unit sphere angular domain $\mathcal{S}$ is:
  \begin{multline}
    \hat{\Omega}\cdot\nabla\Psi_g(\vec{r},\hat{\Omega},t) + \Sigma_{t,g}
    \Psi_g(\vec{r},\hat{\Omega},t) =
    \sum^G_{g'=1}\int_\mathcal{S} \Sigma_s^{g'\rightarrow g}(\hat{\Omega}'\rightarrow\hat{\Omega})
    \Psi_{g'}(\vec{r},\hat{\Omega}',t)d\hat{\Omega}' \\
    + \frac{1}{4\pi}\frac{\chi_{p,g}(1-\beta)}{k}\sum^G_{g'=1} \nu\Sigma_{f,g'} \phi_{g'}(\vec{r},t)
    + \frac{1}{4\pi}\sum^I_{i=1}\chi_{d,g}
    \lambda_i C_i(\vec{r},t)
    \label{eq:mg-nte}
  \end{multline}
  with boundary conditions:
  %
  \begin{gather}
    \Psi_g(\vec{r},\hat{\Omega}) = \Psi^\text{inc}_g(\vec{r},\hat{\Omega}) +
    \alpha^s_g\Psi_g(\vec{r},\hat{\Omega}_r)
    \mbox{ on } \vec{r} \in \partial\mathcal{D} \mbox{ and } \hat{\Omega}\cdot\hat{n}_b < 0,
    \shortintertext{where}
    \begin{align*}
%      \chi_{p,g} &= \mbox{prompt fission neutron spectrum in group $g$,} \\
%      \beta &= \sum^I_{i=1} \beta_i = \mbox{total delayed neutron fraction,} \\
%      \chi_{d,g} &= \mbox{delayed fission neutron spectrum in group $g$,} \\
%      \lambda_i &= \mbox{decay constant of precursor group $i$,} \\
%      C_i &= \mbox{delayed neutron precursor concentration for group $i$,} \\
      \Psi^\text{inc}_g &= \mbox{incident surface source in group $g$,} \\
      \alpha^s_g &= \mbox{specular reflectivity on }\partial \mathcal{D} \mbox{ for group }g, \\
      \hat{\Omega}_r &= \hat{\Omega}-2(\hat{\Omega}\cdot \hat{n}_b)\hat{n}_b = \mbox{reflection angle}, \\
      \hat{n}_b &= \mbox{outward unit normal vector on the boundary.}
    \end{align*}
  \end{gather}
\end{frame}

\begin{frame}
  \frametitle{Hybrid $S_N$-Diffusion Method}
  \textbf{Self-Adjoint Angular Flux (SAAF) Formulation of the Multigroup $\bm{S_N}$ Equations}
  \vspace{.2cm}

  Second-order linear neutron transport equation derived by obtaining the analytical solution of
  angular flux and substituting it back into the $S_N$ equations.

  \begin{block}{\textbf{Implementation Details}}
    \begin{itemize}
      \item Implemented with finite element method (FEM)
      \item Compatible with the efficient and scalable Hypre-BoomerAMG preconditioning
      \item Uses a modified formulation to handle $1/\Sigma_{t,g}$ factor in near-void regions (similar to
        Streamline-Upwind/Petrov Galerkin (SUPG) stabilization scheme) \cite{wang_diffusion_2014}
      \item Level-symmetric quadrature set for angular discretization (up to $S_{18}$)
      \item Nonlinear diffusion acceleration scheme \cite{wang_diffusion_2014}
    \end{itemize}
  \end{block}
\end{frame}

\begin{frame}
  \frametitle{Hybrid $S_N$-Diffusion Method}
  \textbf{Multigroup Neutron Diffusion Equations}
  \begin{gather}
    - \nabla \cdot D_g \nabla \phi_g + \Sigma^r_g \phi_g =
    \sum^G_{g' \neq g} \Sigma^s_{g' \rightarrow g} \phi_{g'}
    + \frac{\chi_{p,g} \left( 1-\beta \right)}{k} \sum^G_{g'=1} \nu \Sigma^f_{g'}
    \phi_{g'} + \chi^d_g \sum^I_i \lambda_i C_i \label{eq:neutron} %\\
  \end{gather}
  %
  Traditionally, $D_g$ is determined through region-wide neutron interaction tallies in
  high-fidelity neutron transport simulations as follows:
  \begin{align}
    D_g =& \frac{1}{3\Sigma_{t,g}} \quad \mbox{(isotropic)} \\
    D_g =& \frac{1}{3\Sigma_{tr,g}} = \frac{1}{3\left(\Sigma_{t,g}-
    \Sigma_{s1,g}\right)}
    \quad \mbox{(linearly anisotropic)} \label{eq:p1-diffcoef}
    \shortintertext{where}
    \Sigma_{tr} =& \mbox{ macroscopic transport cross section} \nonumber
  \end{align}
\end{frame}

\begin{frame}
  \frametitle{Hybrid $S_N$-Diffusion Method}
  \textbf{Diffusion Correction Scheme (Implemented in Python)}
  \vspace{.2cm}

  In this work, I investigated two transport correction schemes.
  The first scheme is the diffusion
  correction scheme similar to Fick's first law of diffusion:
  \begin{gather}
    D^s_g(x) = -J^{tr}_g(x)\bigg/\frac{d\phi^{tr}_g(x)}{dx} \label{eq:svdc}
  \end{gather}
  %
  where $D^s_g$ is the diffusion correction parameter, and the $tr$ superscript denotes the
  transport-derived neutron current and scalar flux solutions from the $S_N$ method.
  \vspace{.2cm}

%  $D^s_g$ provides pointwise corrections to closely match the diffusion flux solution to the $S_N$
%  flux solution.
%  By replacing $D_g$ with $D^s_g$, we are effectively adding the following transport correction term
%  %
%  \begin{gather}
%    -\nabla\cdot (D^s_g-D_g)\nabla\phi_g
%  \end{gather}
%  to the neutron diffusion equations.
%  \vspace{.2cm}

  $\Rightarrow$ Abandoned after 1-D investigations due to division-by-zero issues near flux maximas
  and minimas.
\end{frame}

\begin{frame}
  \frametitle{Hybrid $S_N$-Diffusion Method}
  \textbf{Drift Correction Scheme (Implemented on Moltres)}
  \vspace{.2cm}

  The second scheme is the drift correction scheme arising from adding a drift term
  ($\vec{D}_g\cdot\nabla \phi_g$) \cite{wang_diffusion_2014} to the neutron diffusion equations:
  \begin{gather}
  \scalebox{0.95}{$
    \vec{D}_g = \frac{\sum^{N_d}_{d=1}w_d\left(\tau_g\hat{\Omega}_d\hat{\Omega}_d\cdot\nabla\Psi_{g,d}
    + \left(\tau_g\Sigma_{t,g}-1\right)\hat{\Omega}_d\Psi_{g,d}
    - \tau_g\sum^G_{g'=1}\Sigma^{g'\rightarrow g}_{s,1}\hat{\Omega}_d\Psi_{g',d}
- D_g\nabla\Psi_{g,d}\right)}{\sum^{N_d}_{d=1}w_d\Psi_{g,d}},$} \label{eq:drift} \\
  \scalebox{0.95}{$
    \gamma_g =
    \frac{\sum_{\hat{\Omega}_d\cdot\hat{n}_b > 0}w_d |\hat{\Omega}_d\cdot\hat{n}_b |
  \Psi_{g,d}}{\sum^{N_d}_{d=1}w_d\Psi_{g,d}}.$} \label{eq:bound-coef}
  \end{gather}
  \vspace{.2cm}

  The drift term also provides pointwise corrections to the neutron diffusion equations. This
  formulation is derived from the SAAF-$S_N$ equations by integrating over the angular domain and
  eliminating terms shared by the neutron diffusion equations.
\end{frame}

\begin{frame}
  \frametitle{Hybrid $S_N$-Diffusion Method}
  \textbf{$\bm{S_N}$ Subsolver Boundary Conditions}
  \vspace{.2cm}

  For the hybrid $S_N$-diffusion method to converge, it requires appropriate boundary conditions for
  the $S_N$ subproblem.

  The $P_1$ approximation for evaluating the neutron angular flux along
  the discrete ordinates $\hat{\Omega}_d$ of the $S_N$ angular quadrature set is:
  \begin{align}
    \Psi_{g,d} &\approx \frac{1}{4\pi}\left(\phi^\text{diff}_g+3\hat{\Omega}_d\cdot
    \vec{J}^\text{diff}_g\right) \nonumber \\
    &=\frac{1}{4\pi}\left(\phi^\text{diff}_g-3\hat{\Omega}_d\cdot D_g\nabla\phi^\text{diff}_g\right)
  \end{align}
  Therefore, the boundary source term for the $S_N$ subsolver is:
  \begin{gather}
    \Psi^\text{inc}_{g,d} = \frac{1}{w}
    \left(\phi^\text{diff}_g-3\hat{\Omega}_d\cdot D_g\nabla\phi^\text{diff}_g\right)
  \end{gather}
  where $w$ is the sum of weights of the level-symmetric quadrature set.
\end{frame}

\begin{frame}
%  \frametitle{Hybrid $S_N$-Diffusion Method Algorithm}
  \begin{columns}
    \column{.35\textwidth}
    \vspace{1cm}

  {\small
    \textbf{Hybrid $\bm{S_N}$-Diffusion Method Algorithm}
    \vspace{.2cm}

    Legend:
    \vspace{.1cm}

  $V_0$: Full problem domain
  \vspace{.1cm}

  $V_1$: Problem subdomain containing control rod region
  \vspace{.1cm}

  $D^{s,(i)}_g$: diffusion correction parameter in the $i$-th iteration
  \vspace{.1cm}

  $\vec{D}^{(i)}_g$: drift correction parameter in the $i$-th iteration
\vspace{4cm}}
  \column{.65\textwidth}
  \begin{figure}
    \centering
    \includegraphics[width=.53\textwidth]{images/algorithm}
    \begin{minipage}[b]{.49\textwidth}
      \caption{Algorithm flowchart for the hybrid $S_N$-diffusion method.}
    \end{minipage}
  \end{figure}
\end{columns}
\end{frame}

\begin{frame}
  \frametitle{Correction Region ($V_1$) and Buffer Region}

  \begin{columns}
    \column{5.5cm}
    \begin{figure}[htb!]
      \centering
      \includegraphics[width=\columnwidth]{hybrid-illustration}
      \caption{1-D geometry for Case 3b.}
      \label{fig:3b-geometry}
    \end{figure}
    \column{5.5cm}
    \begin{itemize}
      \item The approximate $S_N$ boundary conditions will yield some flux deviations near the correction
    region boundary.
      \item This affects transport correction parameters near the boundary.
    \end{itemize}
  \end{columns}
  \begin{figure}[htb!]
      \centering
      \begin{subfigure}[t]{.46\textwidth}
          \centering
          \includegraphics[width=.9\textwidth]{case-3b-group-1-drift}
      \end{subfigure}
      \hfill
      \begin{subfigure}[t]{.46\textwidth}
          \centering
          \includegraphics[width=.9\textwidth]{case-3b-group-8-drift}
      \end{subfigure}
      \caption{The reference and hybrid drift ($\vec{D}_g$) distributions for group 1 and 8 calculated
        from $S_8$ and $S_8$-diffusion simulations. The correction subregion $V_1$ spans $x=0$ cm to
        $x=10$ cm.}
      \label{fig:3b-drift-1}
  \end{figure}
\end{frame}

\begin{frame}
  \frametitle{Correction Region ($V_1$) and Buffer Region}

  A natural/intuitive criterion for the location of the buffer region cutoff boundary
  would be wherever the components of the drift correction variable $\vec{D}_g$ is zero, i.e.,
  wherever the components change signs.
  \begin{enumerate}
    \item At points where the $\vec{D}_g$ components are zero, the flux is approximately isotropic
      along the axes corresponding to the components.
    \item This choice preserves the smoothness of the neutron flux gradient.
  \end{enumerate}
  \begin{figure}[htb!]
      \centering
      \begin{subfigure}[t]{.46\textwidth}
          \centering
          \includegraphics[width=.9\textwidth]{case-3b-group-1-drift}
      \end{subfigure}
      \hfill
      \begin{subfigure}[t]{.46\textwidth}
          \centering
          \includegraphics[width=.9\textwidth]{case-3b-group-8-drift}
      \end{subfigure}
      \caption{The reference and hybrid drift ($\vec{D}_g$) distributions for group 1 and 8 calculated
        from $S_8$ and $S_8$-diffusion simulations. The correction subregion $V_1$ spans $x=0$ cm to
        $x=10$ cm.}
      \label{fig:3b-drift-1}
  \end{figure}
\end{frame}

\begin{frame}
  \frametitle{Hybrid $S_N$-Diffusion Method}
  \textbf{Numerical Implementation}
  \vspace{.2cm}

  The SAAF-$S_N$ and hybrid $S_N$-diffusion method with the drift correction scheme were
  implemented on Moltres.
  \begin{itemize}
    \item Preconditioned Jacobian-free Newton-Krylov (PJFNK) solver \cite{knoll_jacobian-free_2004}
    \item Hypre-BoomerAMG (Algebraic multigrid) preconditioning \cite{hypre_hypre_2022}
    \item MultiApp and Transfers systems from MOOSE for iterative coupling and data transfers
    \item Supporting material and utility C++ classes for loading group constant data and performing
      angular quadrature calculations
  \end{itemize}
  \vspace{.2cm}

  A 1-D hybrid $S_N$-diffusion method with the diffusion correction
  scheme was implemented in Python. (This scheme was abandoned after 1-D analyses due to division by
  zero errors wherever the flux gradients approach zero.)
  \begin{itemize}
    \item 1-D $S_N$ method with diamond differencing scheme
    \item 1-D neutron diffusion method with finite differencing scheme
  \end{itemize}
\end{frame}

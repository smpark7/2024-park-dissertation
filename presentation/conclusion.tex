\begin{frame}
  \frametitle{Conclusion}
  \begin{block}{\textbf{Moltres Multiphysics V\&V Studies}}
    \textbf{Relevance}
    \begin{itemize}
      \item Existing multiphysics coupling capabilities in Moltres must be verified and validated.
    \end{itemize}
    \textbf{Results Presented}
    \begin{itemize}
      \item CNRS numerical benchmark study with Moltres (Dissertation)
      \item MSRE pump transient V\&V study with Moltres
    \end{itemize}
    \textbf{Major takeaways}
    \begin{itemize}
      \item Moltres is highly consistent with other MSR simulation tools in accurately modeling:
      \begin{itemize}
        \item Salt flow-induced temperature and DNP drift
        \item Temperature ractivity feedback due to Doppler broadening and thermal expansion
        \item Buoyancy-driven flow due to temperature gradients
        \item Time-varying and looped precursor flow
      \end{itemize}
    \end{itemize}
  \end{block}
\end{frame}

\begin{frame}
  \frametitle{Conclusion}
  \begin{block}{\textbf{Spalart-Allmaras Turbulence Model Implementation in Moltres}}
    \textbf{Relevance}
    \begin{itemize}
      \item Moltres requires turbulence modeling capability to model turbulent salt
        flow in MSRs.
    \end{itemize}
    \textbf{Results Presented}
    \begin{itemize}
      \item Implementation of a Spalart-Allmaras turbulence model in Moltres
      \item V\&V studies of turbulent channel, turbulent pipe, and backward-facing step flow
    \end{itemize}
    \textbf{Major Takeaways}
    \begin{itemize}
      \item The Spalart-Allmaras turbulence model in Moltres is accurate for turbulent channel and
        pipe flows.
      \item It is also consistent with other implementations when simulating the
        backward-facing step flow problem.
    \end{itemize}
  \end{block}
\end{frame}

\begin{frame}
  \frametitle{Conclusion}
  \begin{block}{\textbf{Hybrid $\bm{S_N}$-Diffusion Method}}
    \textbf{Relevance}
    \begin{itemize}
      \item There are no MSR simulation tools or studies that explicitly model moving control rods.
    \end{itemize}
    \textbf{Results Presented}
    \begin{itemize}
      \item Development and implementation of a hybrid $S_N$-diffusion method for
        accurate control rod modeling in Moltres
      \item Verification, validation, and demonstration of the hybrid $S_N$-diffusion
        method through 1-D, 2-D, \& 3-D analyses of the MSRE
    \end{itemize}
    \textbf{Major Takeaways}
    \begin{itemize}
      \item The hybrid method is an iterative method that applies a $S_N$ subsolver around control
        rod regions for improved rod worth estimates.
      \item The hybrid method applies an adaptive boundary coupling between
        the $S_N$ subsolver and the neutron diffusion solver to smooth flux gradients.
      \item The hybrid method accurately reproduces control rod worths at only four times the
        computational cost of the neutron diffusion method.
      \item The hybrid method exhibits good scalability on high-performance computing clusters.
    \end{itemize}
  \end{block}
\end{frame}

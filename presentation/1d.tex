\begin{frame}
  \frametitle{1-D MSRE Neutronics Model Geometries}
  \begin{figure}[htb!]
    \centering
    \includegraphics[width=0.9\columnwidth]{case-geometry}
    \caption{Geometries of the 1-D test cases. The material labeled ``mixture'' represents a
      homogeneous mixture of fuel and graphite at a ratio of 22.5\%-77.5\% by volume. All geometries
      have reflective boundary conditions on the boundary at $x=0$ cm. The right-side boundaries are
      reflective for Cases 1a and 1b, and vacuum for Cases 2a, 2b, 3a, and 3b.}
    \label{fig:case-geom}
  \end{figure}
\end{frame}

\begin{frame}
  \frametitle{1-D MSRE Neutronics Modeling Approach}
  \begin{columns}
    \column{5.5cm}
    \textbf{1-D Neutronics Model Setup}
    \vspace{.2cm}

    \begin{itemize}
      \item Material compositions derived from the MSRE design
      \item Reduced gadolinium content in control rod to 0.35 wt\%
      \item Eight neutron energy groups
      \item Group constants generated using OpenMC with up to 2nd-order Legendre expansions of scattering
        matrices
      \item Uniform temperature at 900 K
    \end{itemize}
    \column{5.5cm}
    \begin{table}[h]
      \centering
      \caption{Neutron energy group structure in this work. Originally devised by Jaradat
      \cite{jaradat_development_2021-1}.}
      \begin{tabular}{r S}
        \toprule
        Group & {Upper energy bound [eV]} \\
        \midrule
        1 & 2.000$\times 10^7$ \\
        2 & 1.353$\times 10^6$ \\
        3 & 6.734$\times 10^4$ \\
        4 & 9.118$\times 10^3$ \\
        5 & 1.487$\times 10^2$ \\
        6 & 4.000$\times 10^0$ \\
        7 & 6.250$\times 10^{-1}$ \\
        8 & 8.000$\times 10^{-2}$ \\
        \bottomrule
      \end{tabular}
      \label{table:energy-group}
    \end{table}
  \end{columns}
\end{frame}

\begin{frame}
  \frametitle{1-D MSRE Neutronics Modeling Approach}
  \textbf{1-D Neutronics Model Numerical Solvers}
  \vspace{.2cm}

  All 1-D cases ran on each of the following numerical solvers:
  \begin{enumerate}
    \item OpenMC in continuous energy mode (OpenMC-CE)
    \item OpenMC in multigroup mode (OpenMC-MG)
    \item Standard neutron diffusion method (Moltres \& Python)
    \item $S_8$ method (Moltres \& Python)
    \item Hybrid $S_8$-diffusion method (Moltres \& Python)
  \end{enumerate}
  \vspace{.2cm}

  \textbf{Reactivity \& Reactivity Difference}
  \begin{gather}
    \mbox{Reactivity } \rho \equiv \frac{k_\text{eff}-1}{k_\text{eff}}.
  \end{gather}
  \begin{gather}
    \Delta\rho = \rho_1 - \rho_2 =
    \frac{k_{\text{eff},1}-k_{\text{eff},2}}{k_{\text{eff},1}k_{\text{eff},2}}.
  \end{gather}
\end{frame}

\begin{frame}
  \frametitle{1-D MSRE Neutronics Simulation Results}
  \textbf{1-D Neutronics Model Reactivity Results}
  \begin{figure}[h]
    \centering
    \includegraphics[width=\columnwidth]{rho}
    \caption{Difference in reactivity $\rho$ of all neutronics methods investigated relative
    to OpenMC-CE.}
    \label{fig:1d-rho}
  \end{figure}
\end{frame}

\begin{frame}
  \frametitle{1-D MSRE Neutronics Simulation Results}
  \textbf{1-D Neutronics Model Control Rod Worth Results}
  \begin{figure}[h]
    \centering
    \includegraphics[width=\columnwidth]{worth}
    \caption{Percentage difference in rod worth for Cases 2 and 3 of all neutronics methods
    investigated relative to OpenMC-CE.}
    \label{fig:1d-worth}
  \end{figure}
\end{frame}

\begin{frame}
%  \frametitle{Hybrid $S_N$-Diffusion Method: 1-D Neutronics Eigenvalue Simulations}
  \begin{columns}
    \column{5.5cm}
    \textbf{Case 3b Neutron Flux Distributions}
    \begin{itemize}
      \item The neutron diffusion and hybrid methods fare worse than the $S_8$ method at capturing
        the oscillatory flux pattern.
      \item The hybrid method performs better than the neutron diffusion method near $x=0$ cm where
        the control rod is situated.
    \end{itemize}
    \column{5.5cm}
    \begin{figure}[htb!]
      \centering
      \includegraphics[width=.975\columnwidth]{case-3b-flux-diff}
      \caption{Absolute difference in neutron group flux distributions for Case 3b from Moltres-$S_8$,
      Moltres-diffusion, Moltres-hybrid, and Python-hybrid relative to OpenMC-MG.}
      \label{fig:3b-flux-diff}
    \end{figure}
  \end{columns}
\end{frame}

\begin{frame}
  \frametitle{1-D MSRE Neutronics Simulation Results}
  \textbf{Impact of Correction Subregion Sizes on Rod Worth}
  \begin{itemize}
    \item Rod worth estimates vary non-monotonically with increasing correction subregion size.
    \item Rod worth estimates remain within 0.2 \% of the $S_8$ method rod worth.
    \item The hybrid method produces accurate rod worth estimates as long as the correction region
      size is kept consistent.
  \end{itemize}
  \begin{figure}[htb!]
    \centering
    \includegraphics[width=0.6\columnwidth]{correction-size-rho}
    \caption{Percentage difference in rod worth from the hybrid method relative to OpenMC-CE for
      Cases 3a and 3b with different correction subregion sizes. The horizontal lines indicate
      equivalent rod worth differences from the OpenMC-MG and $S_8$ methods.}
    \label{fig:v1-size-rho}
  \end{figure}
\end{frame}

\begin{frame}
  \frametitle{1-D MSRE Neutronics Simulation Results}
  \begin{columns}
    \column{7cm}
    \textbf{Relaxing the $S_N$ Convergence Tolerance}
    \begin{itemize}
      \item Transport correction parameters converge faster than scalar flux in the $S_N$ subsolver.
      \item Relaxing the $S_N$ subsolver convergence tolerance would provide computational savings.
      \item The hybrid method exhibits superlinear ($q=1.333$) convergence in $k$ with respect to
        the $S_8$ convergence tolerance value.
    \end{itemize}
    \begin{table}[h]
      \centering
      \caption{Number of outer iterations in hybrid method calculations of Case 3b for a given set of
      convergence tolerance values imposed on the $S_8$ subsolver.}
      \small
      \setlength\tabcolsep{2pt}
      \begin{tabular}{l S S S S S S}
        \toprule
        $S_8$ subsolver tolerance, $\epsilon_\text{tol}$ & {$10^{-8}$} & {$10^{-7}$} & {$10^{-6}$} & {$10^{-5}$} & {$10^{-4}$} & {$10^{-3}$} \\
        \midrule
        Number of outer iterations & 3 & 3 & 3 & 2 & 2 & 1 \\
        \bottomrule
      \end{tabular}
      \label{table:sn-tol}
    \end{table}
    \column{4cm}
    \begin{figure}[h]
      \centering
      \includegraphics[width=\columnwidth]{sn-tol}
      \caption{$k_\text{eff}$ error estimates of Case 3b for a range of convergence tolerance values
      imposed on the $S_8$ subsolver relative to the reference $k_\text{eff}$ value when
      $\epsilon_\text{tol}=10^{-8}$.}
      \label{fig:sn-tol}
    \end{figure}
  \end{columns}
\end{frame}

Section \ref{sec:summary-nts-mtds} highlights the poor performance of neutron diffusion
methods for calculating neutron fluxes near control rods. Strong neutron absorption in the control
rod region produces a highly anisotropic neutron flux extending some distance outside the control
rod. Neutron transport methods, which retain angular dependence of the neutron flux to various
extents, generally fare better than neutron diffusion methods, which have isotropic diffusion
coefficients. However, neutron transport methods are also generally more computationally expensive,
given the increased dimensionality of the problem from the angular component. Adding an angular
dimension to the existing geometric and neutron energy group dimensions dramatically
increases the problem size and the computational resources necessary to solve the system. Many past
efforts have tried introducing
transport correction techniques to improve neutron flux and multiplication factor estimates in
diffusion-based methods. Other than control rod regions, these techniques may also correct
homogenization errors introduced by spatial homogenization of fuel assemblies and other
structures within a reactor core. They invariably rely on neutron transport methods to generate
transport corrections in the form of corrected diffusion coefficients
\cite{bretscher_computing_1997, scherer_determination_1976, ronen_accurate_2004,
pounders_diffusion_2009, kavenoky_sph_1978}, boundary conditions \cite{davison_influence_1951,
pellaud_extrapolation_1968, fen_modelling_1992}, Eddington factors, or discontinuity factors
\cite{koebke_new_1980}.

In this chapter, I propose a hybrid $S_N$-diffusion neutronics method to improve control rod
modeling in neutron diffusion solvers without spatial homogenization. In essence, the hybrid
method is an iterative method that applies
the $S_N$ discrete ordinates neutron transport method on subregions containing the control rod to
obtain pointwise transport corrections for the diffusion method on the subregions.
This chapter presents the mathematical derivation for the
\gls{SAAF} formulation of the $S_N$ equations and the drift transport correction term for the
neutron diffusion equations, and the computational algorithm for the hybrid method.

% Section \ref{sec:theory} discusses the theoretical background for the hybrid $S_N$-diffusion
% method. Section \ref{sec:implementation} provides numerical implementation details of
% the hybrid method and its components. Sections \ref{sec:test-case} and \ref{sec:sim-param} describe
% the 1-D test cases and several simulation parameters in that context. Section
% \ref{sec:prelim-results} discusses the results of the hybrid method applied to the 1-D test cases
% with comparisons to higher-fidelity Monte Carlo and $S_N$ neutron transport methods. Lastly,
% Section \ref{sec:hybrid-summary} summarizes the key findings in this chapter.

% The proposed hybrid $S_N$-diffusion method is an iterative method to improve the
% accuracy of neutron diffusion solutions in reactor systems with highly anisotropic flux regions
% which I will use to model control rods. 

\section{\Gls{SAAF} Formulation of the $S_N$ Method}

\subsection{Multigroup Neutron Transport Equations}

Continuing from the introduction to neutronics methods in Section \ref{sec:summary-nts-mtds},
the time-dependent, multigroup neutron transport equation defined on the 3-D spatial domain
$\mathcal{D}$ and 2-D unit sphere angular domain $\mathcal{S}$ is:
%
\begin{multline}
  \frac{\partial}{\partial t}\left[\frac{\Psi_g(\vec{r},\hat{\Omega},t)}{v_g}\right] +
  \hat{\Omega}\cdot\nabla\Psi_g(\vec{r},\hat{\Omega},t) + \Sigma_{t,g}
  \Psi_g(\vec{r},\hat{\Omega},t) =
  \sum^G_{g'=1}\int_\mathcal{S} \Sigma_s^{g'\rightarrow g}(\hat{\Omega}'\rightarrow\hat{\Omega})
  \Psi_{g'}(\vec{r},\hat{\Omega}',t)d\hat{\Omega}' \\
  + \frac{1}{4\pi}\chi_{p,g}(1-\beta)\sum^G_{g'=1} \nu\Sigma_{f,g'} \phi_{g'}(\vec{r},t)
  + \frac{1}{4\pi}\sum^I_{i=1}\chi_{d,g}
  \lambda_i C_i(\vec{r},t) + Q^{\text{ext}}_g(\vec{r},\hat{\Omega})
  \label{eq:mg-nte}
\end{multline}
%
with the boundary conditions
%
\begin{gather}
  \Psi_g(\vec{r},\hat{\Omega}) = \Psi^\text{inc}_g(\vec{r},\hat{\Omega}) +
  \alpha^s_g\Psi_g(\vec{r},\hat{\Omega}_r)
  \mbox{ on } \vec{r} \in \partial\mathcal{D} \mbox{ and } \hat{\Omega}\cdot\hat{n}_b < 0,
  \shortintertext{where}
  \begin{align*}
    \chi_{p,g} &= \mbox{prompt fission neutron spectrum in group $g$,} \\
    \beta &= \sum^I_{i=1} \beta_i = \mbox{total delayed neutron fraction,} \\
    \chi_{d,g} &= \mbox{delayed fission neutron spectrum in group $g$,} \\
    \lambda_i &= \mbox{decay constant of precursor group $i$,} \\
    C_i &= \mbox{delayed neutron precursor concentration for group $i$,} \\
    \Psi^\text{inc}_g &= \mbox{incident surface source in group $g$,} \\
    \alpha^s_g &= \mbox{specular reflectivity on }\partial \mathcal{D} \mbox{ for group }g, \\
    \hat{\Omega}_r &= \hat{\Omega}-2(\hat{\Omega}\cdot \hat{n}_b)\hat{n}_b, \\
    \hat{n}_b &= \mbox{outward unit normal vector on the boundary.}
  \end{align*}
\end{gather}
%
In order to introduce operators and facilitate subsequent mathematical derivations, we will define
the following vector forms:
%
\begin{gather}
  \bm{\Psi} \equiv
  \begin{bmatrix}
    \Psi_1 \\
    \Psi_2 \\
    \vdots \\
    \Psi_G
  \end{bmatrix},
  \bm{\Phi} \equiv \int_S \bm{\Psi}d\hat{\Omega} \equiv
  \begin{bmatrix}
    \phi_1 \\
    \phi_2 \\
    \vdots \\
    \phi_G
  \end{bmatrix},
  \bm{\frac{\Psi}{v}} \equiv
  \begin{bmatrix}
    \frac{\Psi_1}{v_1} \\
    \frac{\Psi_2}{v_2} \\
    \vdots \\
    \frac{\Psi_G}{v_G}
  \end{bmatrix},
  \bm{C} \equiv
  \begin{bmatrix}
    C_1 \\
    C_2 \\
    \vdots \\
    C_G
  \end{bmatrix},
  \bm{Q}^{\text{ext}} \equiv
  \begin{bmatrix}
    Q^\text{ext}_1 \\
    Q^\text{ext}_2 \\
    \vdots \\
    Q^\text{ext}_G
  \end{bmatrix}, \nonumber
%  \bm{Q}_f = \frac{1}{4\pi}\bm{Q}_{f,0} =
%  \begin{bmatrix}
%    \frac{1}{4\pi}\chi_{p,1}(1-\beta)\sum^G_{g'=1} \nu\Sigma_{f,g'} \phi_{g'} +
%    \frac{1}{4\pi}\sum^I_{i=1}\chi_{d,1} \lambda_i C_i \\
%    \frac{1}{4\pi}\chi_{p,2}(1-\beta)\sum^G_{g'=1} \nu\Sigma_{f,g'} \phi_{g'} +
%    \frac{1}{4\pi}\sum^I_{i=1}\chi_{d,2} \lambda_i C_i \\
%    \vdots \\
%    \frac{1}{4\pi}\chi_{p,G}(1-\beta)\sum^G_{g'=1} \nu\Sigma_{f,g'} \phi_{g'} +
%    \frac{1}{4\pi}\sum^I_{i=1}\chi_{d,G} \lambda_i C_i
%  \end{bmatrix},
%  \bm{Q} = \bm{Q}_f + \bm{Q}^\text{ext}. \nonumber
\end{gather}
%
We also define the following operators:
%
\begin{gather}
  \mathbb{L}_1\bm{\Psi} \equiv
  \begin{bmatrix}
    \hat{\Omega}\cdot\nabla\Psi_1 \\
    \hat{\Omega}\cdot\nabla\Psi_2 \\
    \vdots \\
    \hat{\Omega}\cdot\nabla\Psi_G \\
  \end{bmatrix},
  \mathbb{L}_2\bm{\Psi} \equiv
  \begin{bmatrix}
    \Sigma_{t,1}\Psi_1 \\
    \Sigma_{t,2}\Psi_2 \\
    \vdots \\
    \Sigma_{t,G}\Psi_G
  \end{bmatrix},
  \mathbb{L}\bm{\Psi} \equiv \mathbb{L}_1\bm{\Psi} + \mathbb{L}_2\bm{\Psi}, \nonumber \\
  \mathbb{S}\bm{\Psi} \equiv
  \begin{bmatrix}
    \sum^G_{g'=1}\int_S \Sigma_s^{g'\rightarrow 1}\Psi_{g'}d\hat{\Omega} \\
    \sum^G_{g'=2}\int_S \Sigma_s^{g'\rightarrow 2}\Psi_{g'}d\hat{\Omega} \\
    \vdots \\
    \sum^G_{g'=G}\int_S \Sigma_s^{g'\rightarrow G}\Psi_{g'}d\hat{\Omega}
  \end{bmatrix},
  \mathbb{B}\bm{\Psi} \equiv
  \begin{bmatrix}
    \alpha^s_1\Psi_1(\hat{\Omega}_r) \\
    \alpha^s_2\Psi_2(\hat{\Omega}_r) \\
    \vdots \\
    \alpha^s_G\Psi_G(\hat{\Omega}_r)
  \end{bmatrix}, \nonumber \\
  \mathbb{F}\bm{\Psi} \equiv \frac{1}{4\pi}\mathbb{F}_0\bm{\Psi} \equiv
  \begin{bmatrix}
    \frac{1}{4\pi}\chi_{p,1}(1-\beta)\sum^G_{g'=1}\nu\Sigma_{f,g'}\phi_{g'} \\
    \frac{1}{4\pi}\chi_{p,2}(1-\beta)\sum^G_{g'=1}\nu\Sigma_{f,g'}\phi_{g'} \\
    \vdots \\
    \frac{1}{4\pi}\chi_{p,G}(1-\beta)\sum^G_{g'=1}\nu\Sigma_{f,g'}\phi_{g'}
  \end{bmatrix},
  \mathbb{C}\bm{C} \equiv \frac{1}{4\pi}\mathbb{C}_0\bm{C} \equiv
  \begin{bmatrix}
    \frac{1}{4\pi}\sum^I_{i=1}\chi_{d,1} \lambda_i C_i \\
    \frac{1}{4\pi}\sum^I_{i=1}\chi_{d,2} \lambda_i C_i \\
    \vdots \\
    \frac{1}{4\pi}\sum^I_{i=1}\chi_{d,G} \lambda_i C_i
  \end{bmatrix}. \nonumber
\end{gather}
%
Note that the operators apply element-wise multiplication as opposed to the more conventional
matrix multiplication. Eq.\ \ref{eq:mg-nte} can be reexpressed as:
%
\begin{gather}
  \frac{\partial}{\partial t} \left(\frac{\bm{\Psi}}{\bm{v}}\right)+\mathbb{L}\bm{\Psi}
  = \mathbb{S}\bm{\Psi} + \bm{Q}, \label{eq:nte-vec}
\end{gather}
%
with the boundary conditions on $\partial\mathcal{D}$
%
\begin{gather}
  \bm{\Psi} = \bm{\Psi}^\text{inc} + \mathbb{B}\bm{\Psi},
  \shortintertext{where}
  \begin{align*}
    \bm{Q} &= \bm{Q}_f + \bm{Q}^\text{ext}, & \\
    \bm{Q}_f &= \mathbb{F}\bm{\Psi} + \mathbb{C}\bm{C}. &
  \end{align*}
\end{gather}

\subsection{Weak Form of the Multigroup Neutron Transport Equations}

Before we begin the derivation for the \gls{SAAF} formulation, we introduce inner product notations
to express Eq.\ \ref{eq:nte-vec} and the eventual \gls{SAAF} formulation in their weak forms to be
solved using \gls{FEM}. We define the inner product consisting of volume integrals on
$\mathcal{D}\otimes\mathcal{S}$ as
%
\begin{gather}
  (\bm{a}, \bm{b})_{\mathcal{D}\otimes\mathcal{S}} \equiv
  \sum^G_{g=1}\int_\mathcal{S}d\hat{\Omega}\int_\mathcal{D}d\vec{r}\
  a_g(\vec{r},\hat{\Omega}) b_g(\vec{r},\hat{\Omega}), \label{eq:weak-domain}
\end{gather}
%
where $\bm{a}$ and $\bm{b}$ represent multigroup function vectors such as $\bm{\Psi}$. We also
define boundary integrals on $\partial\mathcal{D}\otimes\mathcal{S}$ as
%
\begin{gather}
  \langle\bm{a},\bm{b}\rangle^\pm_{\partial\mathcal{D}\otimes\mathcal{S}} \equiv
  \sum^G_{g=1}\int_{\partial\mathcal{D}}d\vec{r}
  \int_{\mathcal{S}^{\pm}}d\hat{\Omega}\ |\hat{\Omega}\cdot\hat{n}_b|
  a_g(\vec{r},\hat{\Omega}) b_g(\vec{r},\hat{\Omega}), \label{eq:weak-boundary}
\end{gather}
%
where the $\pm$ sign depends on the signedness of $\hat{\Omega}\cdot\hat{n}_b$ at the boundary.
We henceforth drop the phase space subscripts for brevity.
%
In accordance with standard weak form derivation procedure, we multiply Eq.\ \ref{eq:nte-vec}
by a test function $\bm{\Psi}^*$ and integrate throughout over the $\mathcal{D}$ and $\mathcal{S}$
domains
%
\begin{gather}
  \left(\bm{\Psi}^*,\frac{\partial}{\partial t}\left(\frac{\bm{\Psi}}{\bm{v}}\right)\right) +
  \left(\bm{\Psi}^*,\mathbb{L}\bm{\Psi}\right) = \left(\bm{\Psi}^*,\mathbb{S}\bm{\Psi}\right) +
  \left(\bm{\Psi}^*,\bm{Q}\right).
\end{gather}
%
We apply integration by parts on the streaming term to obtain
%
\begin{gather}
  \left(\bm{\Psi}^*,\frac{\partial}{\partial t}\left(\frac{\bm{\Psi}}{\bm{v}}\right)\right) +
  \left(\mathbb{L}^*\bm{\Psi}^*,\bm{\Psi}\right) + \langle\bm{\Psi}^*,\bm{\Psi}\rangle^+ -
  \langle\bm{\Psi}^*,\bm{\Psi}\rangle^- = \left(\bm{\Psi}^*,\mathbb{S}\bm{\Psi}\right) +
  \left(\bm{\Psi}^*,\bm{Q}\right), \label{eq:nte-weak}
\end{gather}
%
where $\mathbb{L}^*$ is the adjoint of $\mathbb{L}$.
%\begin{multline}
%  \left(\bm{\Psi}^*,\left(\mathbb{I} - \mathbb{L}^{-1}_2\mathbb{L}_1\right)
%  \frac{\partial}{\partial t}\left(\frac{\bm{\Psi}}{\bm{v}}\right)\right) -
%  \left(\mathbb{L}^*_1\bm{\Psi}^*,\mathbb{L}^{-1}_2\mathbb{L}_1\bm{\Psi}\right) +
%  \langle\bm{\Psi}^*,
%  \left(\bm{\Psi}^*,\mathbb{L}_2\bm{\Psi}\right) = \\
%  \left(\bm{\Psi}^*,\left(\mathbb{I} - \mathbb{L}^{-1}_2\mathbb{L}_1\right)\mathbb{S}\bm{\Psi}
%  \right) +
%  \left(\bm{\Psi}^*,\left(\mathbb{I} - \mathbb{L}^{-1}_2\mathbb{L}_1\right) \bm{Q}\right).
%\end{multline}

\subsection{\gls{SAAF} Formulation}

We begin the derivation for the \gls{SAAF} formulation by first rearranging Eq.\ \ref{eq:nte-vec}
as
%
\begin{gather}
  \bm{\Psi} = \mathbb{L}^{-1}_2\left[\mathbb{S}\bm{\Psi}+\bm{Q}
  -\frac{\partial}{\partial t}\left(\frac{\bm{\Psi}}{\bm{v}}\right)-\mathbb{L}_1\bm{\Psi}\right]
  \label{eq:afe}
  \shortintertext{where}
  \mathbb{L}^{-1}_2 =
  \begin{bmatrix}
    \frac{1}{\Sigma_{t,1}} \\
    \frac{1}{\Sigma_{t,2}} \\
    \vdots \\
    \frac{1}{\Sigma_{t,G}}
  \end{bmatrix}. \nonumber
\end{gather}
%
We substitute Eq.\ \ref{eq:afe} back into the streaming term in Eq.\ \ref{eq:nte-weak} and
rearrange the terms to obtain
%
\begin{multline}
  \left(\mathbb{L}^{-1}_2\mathbb{L}\bm{\Psi}^*,\frac{\partial}{\partial t}\left(\frac{\bm{\Psi}}
      {\bm{v}}\right)\right) + 
  \left(\mathbb{L}_1\bm{\Psi}^*,\mathbb{L}^{-1}_2\mathbb{L}_1\bm{\Psi}\right) +
  \langle\bm{\Psi}^*,\bm{\Psi}\rangle^+ - \langle\bm{\Psi}^*,\bm{\Psi}\rangle^- +
  \left(\mathbb{L}_2\bm{\Psi}^*,\bm{\Psi}\right) = \\
  \left(\mathbb{L}^{-1}_2\mathbb{L}\bm{\Psi}^*,
  \mathbb{S}\bm{\Psi}\right) + \left(\mathbb{L}^{-1}_2\mathbb{L}\bm{\Psi}^*,\bm{Q}\right)
  \label{eq:saaf}
\end{multline}
%
using the following relations
%
\begin{gather}
  \mathbb{L}_1\mathbb{L}_2 = \mathbb{L}_2\mathbb{L}_1, \hspace{1cm}
  \mathbb{L}^*_1 = -\mathbb{L}_1,\hspace{1cm}
  \mathbb{L}^*_2 = \mathbb{L}_2, \nonumber \\
  \mathbb{I}-\mathbb{L}^{-1}_2\mathbb{L}^*_1 =
  \mathbb{L}^{-1}_2\mathbb{L}_2 + \mathbb{L}^{-1}_2\mathbb{L}_1 =
  \mathbb{L}^{-1}_2\mathbb{L}. \nonumber
\end{gather}

Notably, the \gls{SAAF} formulation contains a second-order derivative streaming term
(prior to applying integration by parts) which lends itself better to standard
\gls{FEM} solver routines, such as the \gls{GMRES} method, than the first-order streaming
term of the original formulation.
However, the $\mathbb{L}^{-1}_2$ operators render the current \gls{SAAF} formulation
unsolveable in void and near-void regions where the total cross sections $\Sigma_{t,g}$ approach
zero \cite{wang_diffusion_2014}.

\subsection{Spatial Discretization and Void Treatment}

To overcome the issue of voids, Wang et al.\ \cite{wang_diffusion_2014} modified the standard
\gls{SAAF} formulation by taking inspiration from the \gls{SUPG} stabilization technique
\cite{brooks_streamline_1982}. We start by first applying spatial discretization on the \gls{SAAF}
formulation. For \gls{FEM}, the spatial domain is discretized into finite, non-overlapping mesh
elements $\mathcal{D}_e$ such that $\mathcal{D} = \cup_{e\in\mathbb{T}_h}\mathcal{D}_e$, where
$\mathbb{T}_h$ denotes the index set of elements discretizing $\mathcal{D}$ and $\mathcal{D}_e$
is the subdomain covered by the element of index $e$. Similarly, the outermost boundary
$\partial\mathcal{D}$ is discretized by non-overlapping mesh sides indexed by $s$. Within each mesh
element, any spatially varying function can be approximated with basis functions
%
\begin{gather}
  f(\vec{r}) = \sum^N_{i=1}f_i b_i(\vec{r}),
  \shortintertext{where}
  \begin{align*}
    N &= \mbox{degrees of freedom in the mesh element,} \\
    b_i &= \mbox{basis function,} \\
    f_i &= \mbox{coefficient corresponding to $b_i$ in approximating $f$}.
  \end{align*}
\end{gather}

Thus, we can define the inner products
%
\begin{gather}
  \left(\bm{a},\bm{b}\right) \equiv \sum^G_{g=1}\int_\mathcal{S}d\hat{\Omega}
  \sum_{e\in\mathbb{T}_h}\int_{\mathcal{D}_e}d\vec{r}\sum_{i\in E_e}a_{g,i}(\hat{\Omega})
  b_i(\vec{r})\sum_{j\in E_e}b_{g,j}(\hat{\Omega})b_j(\vec{r}), \\
  \langle\bm{a},\bm{b}\rangle^{\pm} \equiv \sum^G_{g=1}\sum_{s\in\partial\mathcal{D}}\int_s
  d\vec{r} \int_{mathcal{S}^\pm_{\hat{n}_b}}d\hat{\Omega} |\hat{\Omega}\cdot\hat{n}_b|
  \sum_{i\in E_s}a_{g,i}(\hat{\Omega})b_i(\vec{r})\sum_{j\in E_s}b_{g,j}(\hat{\Omega})b_j(\vec{r}),
  \shortintertext{where}
  \begin{align*}
    E_e &= \mbox{index set of the local basis functions in element $e$,} \\
    E_s &= \mbox{index set of the local basis functions in element side $s$,}
  \end{align*}
\end{gather}
%
to apply spatial discretization on Eq.\ \ref{eq:saaf} by replacing the preexisting inner product
notations defined in Eqs.\ \ref{eq:weak-domain} and \ref{eq:weak-boundary}.

We can now proceed with the void treatment for stabilization.
We start by defining the following stabilization operator
%
\begin{gather}
  \bm{\tau} \equiv
  \begin{bmatrix}
    \tau_1 \\
    \tau_2 \\
    \vdots \\
    \tau_G
  \end{bmatrix}
  \shortintertext{where}
  \begin{align*}
    \tau_g &=
    \begin{cases}
      \frac{1}{\Sigma_{t,g}} \mbox{ for } h\Sigma_{t,g} \geq \varsigma \\
      \frac{h}{\varsigma} \mbox{ for } h\Sigma_{t,g} < \varsigma
    \end{cases}, & \\
    h &= \mbox{mesh element size,} & \\
    \varsigma &= \mbox{void constant.} &
  \end{align*}
\end{gather}
%
Next, we redefine Eq.\ \ref{eq:afe} as
%
\begin{gather}
  \bm{\Psi} = \left(\mathbb{I}-\bm{\tau}\mathbb{L}_2\right)\bm{\Psi}+
  \bm{\tau}\left[\mathbb{S}\bm{\Psi}+\bm{Q}
  -\frac{\partial}{\partial t}\left(\frac{\bm{\Psi}}{\bm{v}}\right)-\mathbb{L}_1\bm{\Psi}\right]
  \label{eq:afe-vt}
\end{gather}
%
Substituting Eq.\ \ref{eq:afe-vt} into Eq.\ \ref{eq:nte-weak} yields the stabilized \gls{SAAF}
formulation given as
%
\begin{multline}
  \left(\left(\mathbb{I}+\bm{\tau}\mathbb{L}_1\right)\bm{\Psi}^*,
  \frac{\partial}{\partial t}\left(\frac{\bm{\Psi}}{\bm{v}}\right)\right) + 
  \left(\mathbb{L}_1\bm{\Psi}^*,
  \left(\bm{\tau}\mathbb{L}_1-\mathbb{I}+\bm{\tau}\mathbb{L}_2\right)\bm{\Psi}\right) +
  \langle\bm{\Psi}^*,\bm{\Psi}\rangle^+ - \langle\bm{\Psi}^*,\bm{\Psi}\rangle^- +
  \left(\mathbb{L}_2\bm{\Psi}^*,\bm{\Psi}\right) \\
  = \left(\left(\mathbb{I}+\bm{\tau}\mathbb{L}_1\right)\bm{\Psi}^*,\mathbb{S}\bm{\Psi}\right) +
  \left(\left(\mathbb{I}+\bm{\tau}\mathbb{L}_1\right)\bm{\Psi}^*,\bm{Q}\right)
  \label{eq:saaf-vt}
\end{multline}
%
In non-void regions where $\Sigma_{t,g}$ is non-zero, we can disable the stabilization scheme by
setting $\varsigma$ to zero. Then $\bm{\tau}\mathbb{L}=\mathbb{I}$
and the stabilized \gls{SAAF} formulation reduces to the unstabilized formulation in Eq.\
\ref{eq:saaf}.

\subsection{Angular Discretization}

The $S_N$ method applies a collocation method for the angular discretization by solving for the
angular flux along $N_d$ discrete directions $\hat{\Omega}_d$ with weights $w_d$. Numerous options
exist for the choice of the angular quadrature set
$\{\hat{\Omega}_d,w_d\ |\ d=1,\dots,N_d\}$.
For this work, we chose the commonly used level-symmetric quadrature set
\cite{wang_rattlesnake_2018} for ease of implementation, particularly when applying reflective
boundary conditions due to 90-degree rotational symmetry about each of the three Cartesian axes.

We use the level-symmetric quadrature sets to approximate integrals of the angular flux over the
angular domain
%
\begin{gather}
  \phi_g = \int_\mathcal{S} \Psi_{g,d}\hat{\Omega} \approx \sum^{N_d}_{d=1}w_d
  \Psi_{g,d}, \label{eq:0th-mmt}
  \shortintertext{where}
  \Psi_{g,d} = \Psi_g(\hat{\Omega}_d), \nonumber \\
  w = \sum^{N_d}_{d=1} w_d\ (= 8 \mbox{ for 3-D level-symmetric set}). \nonumber
\end{gather}
%
We calculate the scalar flux $\phi_g$ by taking the zeroth moment with respect to $\hat{\Omega}$ as
shown in Eq.\ \ref{eq:0th-mmt}. We can also calculate higher moments
%
\begin{gather}
  \phi_{g,l,m} = \int_\mathcal{S} Y_{l,m}(\hat{\Omega})\Psi_g(\hat{\Omega})d\hat{\Omega} \approx
  \sum^{N_d}_{d=1}w_d Y_{l,m}(\hat{\Omega}_d)\Psi_{g}(\hat{\Omega}_d) \label{eq:nth-mmt}
  \shortintertext{where}
  \begin{align*}
    Y_{l,m}(\mu,\omega) &=
    \begin{cases}
      \sqrt{2}C^{|m|}_l P^{|m|}_l \cos(|m|\omega), & \mbox{ for } 0<m\leq l \\
      C^0_l P^0_l(\mu), & \mbox{ for } m=0, 0\leq l < \infty \\
      \sqrt{2} C^{|m|}_l P^{|m|}_l(\mu) \sin(|m|\omega), & -l \leq m < 0
    \end{cases}, \\
    \mu &= \mbox{cosine of polar angle,} \\
    \omega &= \mbox{azimuthal angle,} \\
    C^{|m|}_l &= \sqrt{\frac{\left(l-|m|\right)!}{\left(l+|m|\right)!}}, \\
    P^m_l &= \mbox{associated Legendre polynomial of degree $l$ and order $m$.}
  \end{align*}
\end{gather}
%
for anisotropic scattering treatment. 

\subsection{Final Forms of the \gls{SAAF} $S_N$ Equations}

Finally, we present the expanded forms of each term in the \gls{SAAF} formulation of the multigroup
$S_N$ neutron transport equations. With the angular discretization scheme laid out, we define the
inner products
%
\begin{gather}
  \left(\bm{a},\bm{b}\right)_\mathcal{D} \equiv \sum^G_{g=1} \int_\mathcal{D}d\vec{r}
  \sum_{i\in E_e}a_{g,i}b_i(\vec{r})\sum_{j\in E_e}b_{g,j}b_j(\vec{r}), \\
  \left(\bm{a},\bm{b}\right)_{\partial\mathcal{D}} \equiv \sum^G_{g=1}
  \sum_{s\in\partial\mathcal{D}}\int_s d\vec{r}\sum_{i\in E_s}a_{g,i}b_i(\vec{r})\sum_{j\in E_s}
  b_{g,j}b_j(\vec{r}),
\end{gather}
%
involving integrals over the spatial domain to explicitly show the $S_N$ angular discretization.

\noindent Time derivative term:
%
\begin{gather}
  \left(\left(\mathbb{I}+\bm{\tau}\mathbb{L}_1\right)\bm{\Psi}^*,
  \frac{\partial}{\partial t}\left(\frac{\bm{\Psi}}{\bm{v}}\right)\right) =
  \sum^G_{g=1}\sum^{N_d}_{d=1}w_d\left(\Psi^*_{g,d}+\tau_g\hat{\Omega}_d\cdot\Psi^*_{g,d},
  \frac{\Psi_{g,d}}{v_g}\right)_\mathcal{D}
\end{gather}
%
Streaming term:
%
\begin{gather}
  \left(\mathbb{L}_1\bm{\Psi}^*,
  \left(\bm{\tau}\mathbb{L}_1-\mathbb{I}+\bm{\tau}\mathbb{L}_2\right)\bm{\Psi}\right) =
  \sum^G_{g=1}\sum^{N_d}_{d=1}w_d\left(\hat{\Omega}_d\cdot\nabla\Psi^*_{g,d},\tau_g\hat{\Omega}
  \cdot\nabla\Psi_{g,d}-(1-\tau_g\Sigma_{t,g})\Psi_{g,d}\right)_\mathcal{D}
\end{gather}
%
Collision term:
%
\begin{gather}
  \left(\mathbb{L}_2\bm{\Psi}^*,\bm{\Psi}\right) =
  \sum^G_{g=1}\sum^{N_d}_{d=1}w_d\left(\Psi^*_{g,d},\Sigma_{t,g}\Psi_{g,d}\right)_\mathcal{D}
\end{gather}
%
Scattering term:
%
\begin{gather}
  \left(\left(\mathbb{I}+\bm{\tau}\mathbb{L}_1\right)\bm{\Psi}^*,\mathbb{S}\bm{\Psi}\right) =
  \sum^G_{g=1}\sum^{N_d}_{d=1}w_d\left(\Psi^*_{g,d}+\tau_g\hat{\Omega}_d\cdot\nabla\Psi^*_{g,d},
  \sum^G_{g'=1}\sum^L_{l=0}\Sigma^{g'\rightarrow g}_{s,l}\sum^l_{m=-l}
  \frac{2l+1}{w}Y_{l,m}(\hat{\Omega}_d)\phi_{g',l,m}\right)_\mathcal{D}
\end{gather}
%
Prompt fission source term:
%
\begin{gather}
  \left(\left(\mathbb{I}+\bm{\tau}\mathbb{L}_1\right)\bm{\Psi}^*,\mathbb{F}\bm{\Psi}\right) =
  \sum^G_{g=1}\sum^{N_d}_{d=1}w_d\left(\Psi^*_{g,d}+\tau_g\hat{\Omega}_d\cdot\nabla\Psi^*_{g,d},
  \frac{1}{w}\bar{\chi}_g\sum^G_{g'=1}\nu\Sigma_{f,g'}\phi_{g'}\right)_\mathcal{D}
\end{gather}
%
where $\bar{\chi}_g=\chi_{p,g}\left(1-\beta\right)$ for time-dependent problems with \glspl{DNP}
and $\bar{\chi}_g=\chi_{g}$ for steady-state problems without \glspl{DNP}.

\noindent Delayed neutron source term:
%
\begin{gather}
  \left(\left(\mathbb{I}+\bm{\tau}\mathbb{L}_1\right)\bm{\Psi}^*,\mathbb{C}\bm{C}\right) =
  \sum^G_{g=1}\sum^{N_d}_{d=1}w_d\left(\Psi^*_{g,d}+\tau_g\hat{\Omega}_d\cdot\nabla\Psi^*_{g,d},
  \frac{1}{w}\sum ^I_{i=1}\chi_{d,g}\lambda_i C_i\right)_\mathcal{D}
\end{gather}
%
We scale both prompt fission and delayed neutron source terms by the inverse of the
multiplication factor $\frac{1}{k}$ for $k$-eigenvalue problems.

\noindent Vacuum boundary term:
%
\begin{gather}
  \langle\bm{\Psi}^*,\bm{\Psi}\rangle^+ - \langle\bm{\Psi}^*,\bm{\Psi}^\text{inc}\rangle^- =
  \begin{cases}
    \sum^G_{g=1}\sum^{N_d}_{d=1}w_d\left(\Psi^*_{g,d},
    \hat{\Omega}_d\cdot\hat{n}_b\Psi_{g,d}\right),
    & \hat{\Omega}\cdot\hat{n}_b>0,\vec{r}\in\partial\mathcal{D}_s \\
    \sum^G_{g=1}\sum^{N_d}_{d=1}w_d\left(\Psi^*_{g,d},
    \hat{\Omega}_d\cdot\hat{n}_b\Psi^\text{inc}_{g,d}\right),
    & \hat{\Omega}\cdot\hat{n}_b<0,\vec{r}\in\partial\mathcal{D}_s
  \end{cases},
\end{gather}
%
where $\Psi^\text{inc}_{g,d}$ is zero on a vacuum boundary or positive for a boundary source.

\noindent Reflecting boundary term:
%
\begin{gather}
  \langle\bm{\Psi}^*,\bm{\Psi}\rangle^+ - \langle\bm{\Psi}^*,\mathbb{B}\bm{\Psi}\rangle^- =
  \begin{cases}
    \sum^G_{g=1}\sum^{N_d}_{d=1}w_d\left(\Psi^*_{g,d},
    \hat{\Omega}_d\cdot\hat{n}_b\Psi_{g,d}\right),
    & \hat{\Omega}\cdot\hat{n}_b>0,\vec{r}\in\partial\mathcal{D}_s \\
    \sum^G_{g=1}\sum^{N_d}_{d=1}w_d\left(\Psi^*_{g,d},
    \hat{\Omega}_d\cdot\hat{n}_b\Psi_{g,d_r}\right),
    & \hat{\Omega}\cdot\hat{n}_b<0,\vec{r}\in\partial\mathcal{D}_s
  \end{cases},
  \shortintertext{where}
  \hat{\Omega}_{d_r} = \hat{\Omega}_d - 2(\hat{\Omega}_d\cdot\hat{n}_b)\hat{n}_b. \nonumber
\end{gather}

This concludes the derivation of the \gls{SAAF} formulation of the $S_N$ neutron transport method
for generating transport corrections in the hybrid $S_N$-diffusion method.

\section{Transport Correction Formulations}

The conventional approach for determining diffusion coefficients for the neutron diffusion
equations in each subregion involves
running a high-fidelity neutron transport simulation to tally region-wide estimates of the neutron
transport cross section. The transport cross section formulation is derived from the $P_1$
approximation of the neutron transport equation with isotropic sources \cite{bell_nuclear_1970} as
shown in Eq.\ \ref{eq:p1-diffcoef}.
Essentially, each defined subregion has a constant diffusion coefficient value. However, as
discussed in Section \ref{sec:summary-nts-mtds}, the neutron diffusion
equation is only valid in regions of high scattering-to-removal ratios with, at most, linearly
anisotropic scattering and small flux gradients. These conditions do not hold near or within
control rods, near interfaces of neighboring materials with highly dissimilar neutronic properties,
and in materials with significant scattering contributions from light nuclei.

Transport corrections to allow neutron diffusion methods to reproduce high-fidelity flux solutions
can be introduced by modifying the neutron diffusion equations.
In this work, we explored two options for applying
transport corrections to the neutron diffusion equations: applying
diffusion corrections or adding a drift correction term.

\subsection{Diffusion Correction}

Diffusion corrections involve replacing the diffusion coefficient $D_g$ in the
diffusion term with ``optimal'' diffusion coefficients based on
pointwise transport corrections. The literature review in Section \ref{sec:summary-nts-mtds} covers
two such
examples in the Ronen method by Ronen \cite{ronen_accurate_2004} and the space-dependent diffusion
coefficients by Pounders \& Rahnema \cite{pounders_diffusion_2009}.

For the hybrid $S_N$-diffusion method, we used a formulation that incorporates pointwise
corrections to the neutron diffusion flux solution from the $S_N$-derived flux solution as follows:
%
\begin{align}
  D^s_g(x) &= -J^{tr}_g(x)\bigg/\frac{d\phi^{tr}_g(x)}{dx}. \label{eq:svdc}
\end{align}
%
where $D^s_g$ is the \gls{SVDC}, and the $tr$ superscript denotes the transport-derived neutron
current and scalar flux solutions from the $S_N$ method. Transport corrections introduced through
$J_{tr}$ are scaled by the flux gradient. We assumed that it varies continuously and is
at least once differentiable except at dissimilar material interfaces. \glspl{SVDC} provide
pointwise corrections to closely match the diffusion flux solution to the $S_N$ flux solution.
By replacing $D_g$ with $D^s_g$, we are effectively adding the following transport correction term
%
\begin{gather}
  -\frac{\partial}{\partial x}(D^s_g-D_g)\frac{\partial\phi_g}{\partial x}
\end{gather}
to the neutron diffusion equations. Alternatively, we may define
$\partial D\equiv\left(D^s_g-D_g)\right)$. This alternative form shows that diffusion
correction applies a multiplicative closure that scales with the flux gradient.

Eq.\ \ref{eq:svdc} is identical to Ronen's \cite{ronen_accurate_2004} and Pounders \& Rahnema's
\cite{pounders_diffusion_2009} formulations for space-dependent diffusion coefficients in Eq.\
\ref{eq:ronen} and Eq.\ \ref{eq:emp}, respectively. In comparison with
the Ronen method, our approaches differ in how $\phi_{tr}$ and $J_{tr}$ are obtained. Starting with
a standard neutron diffusion calculation, the Ronen method applies an analytically derived
transport operator on every iteration to calculate a new estimate of $J_{tr}$ using the $\phi$
solution from the previous iteration. The main difficulty for the Ronen method lies in deriving the
transport operators which has been demonstrated for only 1-D geometries thus far.

For the empirical method developed by Pounders \& Rahnema, they assumed a priori knowledge of the
reference flux solution and used it to generate piecewise-constant, space-dependent diffusion
coefficients.
Pounders \& Rahnema \cite{pounders_diffusion_2009} demonstrated the effectiveness of applying
pointwise corrections derived from analytical or Monte Carlo reference flux solutions. Compared
with conventional $P_1$-based out-scatter and flux-limited approximations of the diffusion
coefficient, their diffusion coefficients showed superior agreement
with the reference flux solutions. They recognized that volume averaging within each mesh element
for piecewise-constant diffusion coefficients introduces some truncation error if the flux is
non-linear within the mesh element. This design choice may be due to an intention to retain the
diffusion coefficient as a constant in the $\frac{d}{dx}D\frac{d\phi}{dx}$ term of the neutron
diffusion equation (Eq.\ \ref{eq:mg-diff}).

Unlike their approach, the \gls{SVDC} formulation in Eq.\ \ref{eq:svdc} allows for continuously
varying diffusion coefficients to reduce truncation error. In practice, the discretization order of
\gls{SVDC} variables in a numerical calculation would be the same as the discretization order of
the reference flux solution. This formulation introduces a minor change to the finite difference
implementation of the 2nd-order diffusion term to handle spatial derivatives of the
diffusion coefficient as $D^s_g$ is not uniform in space.

Regardless of the differences in implementations amongst the methods discussed here,
diffusion corrections applied through Eq.\ \ref{eq:saaf} were
found to be effective in enabling the neutron diffusion method to accurately reproduce reference
flux solutions \cite{gross_comprehensive_2023, pounders_diffusion_2009}.
Howver, a significant challenge for diffusion corrections
involves their resolution near neutron flux peaks. As observed in Eq.\ \ref{eq:svdc}, the neutron
current and flux gradient values do not necessarily reach zero at the same points in space,
resulting in diffusion coefficient values tending to positive or negative infinity when the
flux gradient is close to zero. Pounders \& Rahnema avoided this issue by using
larger mesh sizes to calculate their empirical diffusion coefficients. However, their
remedy contradicts mesh convergence requirements and would worsen flux accuracy in regions with
steep, non-linear fluxes, such as near control rods. For the Ronen method, Gross et al.\
\cite{gross_comprehensive_2023} applied a numerical fix by avoiding correction
calculations wherever the flux gradient values fall below a certain threshold. 

\subsection{Drift Correction Term}

Drift correction terms feature prominently in existing literature as transport correction
terms for \gls{NDA} schemes or nodal diffusion methods
\cite{smith_nodal_1983, smith_assembly_1986, adams_fast_2002, wang_diffusion_2014}. \gls{NDA}
schemes are \gls{HOLO} methods \cite{chacon_multiscale_2017} similar to the \gls{QD} method for
accelerating a neutron transport
method (high-order) with a modified neutron diffusion method (low-order). Drift terms
are added to the low-order neutron diffusion equations to correct them such that the modified
equations can reproduce the transport solution upon reaching iterative convergence. In nodal
diffusion methods, the drift terms are used to rectify the incoming and outgoing neutron
currents between adjacent nodes.

The general form of drift correction terms to be added to the neutron diffusion equations is a
first-order derivative term $\nabla\cdot \vec{D}_g\phi_g$ (note the vector notation). As such,
drift terms apply
multiplicative closures that scale with the flux. This contrasts with diffusion
corrections which scale with the flux gradient instead. Therefore, drift terms do not suffer from
the division-by-zero errors encountered with diffusion corrections.

Derivations for the drift terms depend on the discretization schemes of the high- and low-order
equations. For this work, we will adopt drift term derivations developed by Wang et al.\
\cite{wang_diffusion_2014, wang_rattlesnake_2018}. We start by integrating Eq.\ \ref{eq:saaf-vt}
over the 2-D unit sphere angular domain $\mathcal{S}$ and replacing $\bm{\Psi}^*$ with isotropic
test functions $\bm{\Phi}^*$ to obtain the neutron balance equation:
%
\begin{multline}
  \left(\bm{\Phi}^*,\frac{\partial}{\partial t}\left(\frac{\bm{\Phi}}{\bm{v}}\right)\right)_\mathcal{D}
  + \left(\mathbb{L}_2\bm{\Phi}^*,\bm{\Phi}\right)_\mathcal{D}
  - \left(\nabla\bm{\Phi}^*,\vec{\bm{J}}\right)_\mathcal{D}
  + \left(\bm{\Phi}^*,\bm{J}^\text{out}\right)_{\partial\mathcal{D}_v} + \\
  \left(\bm{\tau}\nabla\bm{\Phi}^*, \frac{\partial}{\partial t}\left(\frac{\vec{\bm{J}}}{\bm{v}}\right)
    + \nabla\cdot(\vec{\vec{\bm{E}}}\bm{\Phi}) +\mathbb{L}_2\vec{\bm{J}} -
    \mathbb{S}_1\vec{\bm{J}}\right)_\mathcal{D}
  = \left(\bm{\Phi}^*,\mathbb{S}_0\bm{\Phi}\right)_\mathcal{D}
  + \left(\bm{\Phi}^*,\bm{Q}_0\right)_\mathcal{D},
\end{multline}
%
\begin{gather}
  \shortintertext{where}
  \vec{\bm{J}} \equiv
  \begin{bmatrix}
    \vec{J}_1 \\
    \vec{J}_2 \\
    \vdots \\
    \vec{J}_G
  \end{bmatrix},
  \bm{J}^\text{out} \equiv
  \begin{bmatrix}
    \int_{|\hat{\Omega}\cdot\hat{n}_b|>0}|\hat{\Omega}\cdot\hat{n}_b|\Psi_1 d\hat{\Omega} \\
    \int_{|\hat{\Omega}\cdot\hat{n}_b|>0}|\hat{\Omega}\cdot\hat{n}_b|\Psi_2 d\hat{\Omega} \\
    \vdots \\
    \int_{|\hat{\Omega}\cdot\hat{n}_b|>0}|\hat{\Omega}\cdot\hat{n}_b|\Psi_G d\hat{\Omega}
  \end{bmatrix},
  \bm{J}^\text{inc} \equiv
  \begin{bmatrix}
    J^\text{inc}_1 \\
    J^\text{inc}_2 \\
    \vdots \\
    J^\text{inc}_G
  \end{bmatrix},
  \vec{\vec{\bm{E}}} \equiv
  \begin{bmatrix}
    \frac{\int_\mathcal{S} \hat{\Omega}\hat{\Omega}\Psi_1 d\hat{\Omega}}{\int_\mathcal{S}\Psi_1 d\hat{\Omega}} \\
    \frac{\int_\mathcal{S} \hat{\Omega}\hat{\Omega}\Psi_2 d\hat{\Omega}}{\int_\mathcal{S}\Psi_2 d\hat{\Omega}} \\
    \vdots \\
    \frac{\int_\mathcal{S} \hat{\Omega}\hat{\Omega}\Psi_G d\hat{\Omega}}{\int_\mathcal{S}\Psi_G d\hat{\Omega}}
  \end{bmatrix}, \nonumber \\
  \mathbb{S}_0 \bm{\Phi} \equiv
  \begin{bmatrix}
    \sum^G_{g'=1}\Sigma_s^{g'\rightarrow 1}\phi_{g'} \\
    \sum^G_{g'=1}\Sigma_s^{g'\rightarrow 2}\phi_{g'} \\
    \vdots \\
    \sum^G_{g'=1}\Sigma_s^{g'\rightarrow G}\phi_{g'}
  \end{bmatrix},
  \mathbb{S}_1 \vec{\bm{J}} \equiv
  \begin{bmatrix}
    \sum^G_{g'=1}\Sigma_{s,1}^{g'\rightarrow 1}\vec{J}_{g'} \\
    \sum^G_{g'=1}\Sigma_{s,1}^{g'\rightarrow 2}\vec{J}_{g'} \\
    \vdots \\
    \sum^G_{g'=1}\Sigma_{s,1}^{g'\rightarrow G}\vec{J}_{g'}
  \end{bmatrix},
  \bm{Q}_0 \equiv
  \begin{bmatrix}
    \int_\mathcal{S} Q_1 d\hat{\Omega} \\
    \int_\mathcal{S} Q_2 d\hat{\Omega} \\
    \vdots \\
    \int_\mathcal{S} Q_G d\hat{\Omega}
  \end{bmatrix}, \nonumber
\end{gather}
%
and $\partial\mathcal{D}_v$ represents sections of $\partial\mathcal{D}$ which are vacuum boundaries. We
assumed that $\bm{Q}$ is an isotropic source because we do not deal with anisotropic sources in
this work outside of scattering.
We define the following expression
%
\begin{gather}
  \vec{\bm{r}}_1 \equiv \frac{\partial}{\partial t}\left(\frac{\vec{\bm{J}}}{\bm{v}}\right)
  + \nabla\cdot(\vec{\vec{\bm{E}}}\bm{\Phi}) + \mathbb{L}_2\vec{\bm{J}} -
  \mathbb{S}_1\vec{\bm{J}} \label{eq:1st-moment}
\end{gather}
%
to simplify the balance equation. We note that Eq.\ \ref{eq:1st-moment} is identical to the second
low-order equation of the \gls{QD} method (Eq.\ \ref{eq:qd-low-1st}).
We also introduce the diffusion operator
%
\begin{gather}
  \mathbb{D}\nabla\bm{\Phi} \equiv
  \begin{bmatrix}
    D_1\nabla\phi_1 \\
    D_2\nabla\phi_2 \\
    \vdots \\
    D_G\nabla\phi_G
  \end{bmatrix},
\end{gather}
%
and recognize that under the diffusion approximation, the vacuum boundary condition is given as
%
\begin{gather}
  -\mathbb{D}\nabla\bm{\Phi}\cdot\hat{n}_b = \frac{\bm{\Phi}}{2}, \hspace{1cm} \vec{r}\in\partial
  \mathcal{D}_v.
\end{gather}
%
We can rewrite the balance equation as
%
\begin{multline}
  \left(\bm{\Phi}^*,\frac{\partial}{\partial t}\left(\frac{\bm{\Phi}}{\bm{v}}\right)\right)_\mathcal{D}
  + \left(\nabla\bm{\Phi}^*, \mathbb{D}\nabla\bm{\Phi}\right)_\mathcal{D}
  + \left(\mathbb{L}_2\bm{\Phi}^*,\bm{\Phi}\right)_\mathcal{D}
  + \left(\bm{\Phi}^*,\frac{\bm{\Phi}}{2}\right)_{\mathcal{D}_v}
  + \left(\nabla\bm{\Phi}^*,\bm{\tau}\vec{\bm{r}}_1-\vec{\bm{J}}-\mathcal{D}\bm{\Phi}\right)_\mathcal{D}
  + \\
  \left(\bm{\Phi}^*,\bm{J}^\text{out}-\frac{\bm{\Phi}}{2}\right)_{\partial\mathcal{D}_v}
  = \left(\bm{\Phi}^*,\mathbb{S}_0\bm{\Phi}\right)_\mathcal{D}
  + \left(\bm{\Phi}^*,\bm{Q}_0\right)_\mathcal{D}. \label{eq:modified-diff}
\end{multline}
%
We can observe that Eq.\ \ref{eq:modified-diff} is a modified neutron diffusion equation with
transport corrections provided by the fifth and sixth terms. The fifth term is a drift term with
the drift vector defined as
%
\begin{gather}
  \vec{\bm{D}} \equiv \frac{\bm{\tau}\vec{\bm{r}}_1-\vec{\bm{J}}-\mathbb{D}\nabla\bm{\Phi}}{\bm{\Phi}}
\end{gather}
%
and the sixth term is a boundary correction term with the boundary coefficient vector defined as
%
\begin{gather}
  \bm{\gamma} \equiv \frac{\bm{J}^\text{out}}{\bm{\Phi}}-\frac{1}{2}\bm{I}.
\end{gather}
%
With the $S_N$ angular discretization scheme, the drift and boundary correction vector components
can be evaluated as
%
\begin{gather}
  \vec{D}_g = \frac{\sum^{N_d}_{d=1}w_d\left(\tau_g\hat{\Omega}_d\hat{\Omega}_d\cdot\nabla\Psi_{g,d}
  + \left(\tau_g\Sigma_{t,g}-1\right)\hat{\Omega}_d\Psi_{g,d}
  - \tau_g\sum^G_{g'=1}\Sigma^{g'\rightarrow g}_{s,1}\hat{\Omega}_d\Psi_{g',d}
  - D_g\nabla\Psi_{g,d}\right)}{\sum^{N_d}_{d=1}w_d\Psi_{g,d}}, \label{eq:drift} \\
  \gamma_g =
  \frac{\sum_{\hat{\Omega}_d\cdot\hat{n}_b > 0}w_d |\hat{\Omega}_d\cdot\hat{n}_b |
  \Psi_{g,d}}{\sum^{N_d}_{d=1}w_d\Psi_{g,d}}. \label{eq:bound-coef}
\end{gather}

\section{Hybrid $S_N$-Diffusion Method} \label{sec:hybrid-method}

As described in the previous section, transport correction formulations are commonly used in
iterative acceleration schemes or hybrid methods to couple high-order neutron transport
calculations with low-order neutron diffusion calculations. However, the high-order calculations
remain computationally intensive and limit their applicability to time-independent neutronics
simulations on \gls{HPC} clusters.
In order to reduce the computational cost of the high-level $S_N$ calculation in a reactor
simulation, I propose reducing
the problem domain of the $S_N$ method to a \textit{correction region} containing the control rod
and its vicinity. Consequently, the hybrid $S_N$-diffusion method may retain accurate neutron flux
and current estimates around the control rod region from the $S_N$ method while making significant
computational cost savings by treating most of the reactor geometry with the neutron diffusion
method alone. Henceforth, I will refer to the $S_N$ calculation on the correction
region as the $S_N$ \textit{subproblem} or \textit{sub-solver}. I define the full problem
domain and the correction region as $V_0$ and $V_1$, respectively, where
$V_1\subseteq V_0$. The iterative algorithm for the hybrid $S_N$-diffusion method is as follows:
%
\begin{enumerate}
  \item Start with an initial neutron diffusion calculation in $V_0$ using the standard neutron
    diffusion method.
  \item Pass the neutron diffusion current estimates along
    $\partial V_1$ to the $S_N$ sub-solver to evaluate the boundary conditions for the $S_N$
    subproblem.
  \item With the $S_N$ sub-solver, evaluate transport correction terms in $V_1$ using Eq.
    \ref{eq:svdc} or \ref{eq:drift}.
  \item Pass the transport correction terms to the neutron diffusion solver to apply corrections
    within $V_1$ while continuing to apply the standard neutron diffusion solver
    in the rest of $V_0$.
  \item Start the next iteration by running a neutron diffusion calculation with transport
    corrections in $V_1$.
  \item Repeat Steps 2-6 until convergence is reached by meeting desired convergence tolerance
    values.
\end{enumerate}
%
Figure \ref{fig:algorithm} shows a visual flowchart of the same hybrid method algorithm.

\begin{figure}[b!]
  \tikzstyle{every node}=[font=\small]
  \centering
  \begin{tikzpicture}
    \node (1) [object] {\textbf{Perform initial neutron\\diffusion calculation in $\bm{V_0}$}};
    \node (2) [process, below of=1, yshift=-1.8cm]
      {\textbf{Calculate incident flux boundary conditions along\\$\bm{\partial V_1}$ for $\bm{S_N}$ calculation}};
    \node (3) [process, below of=2, yshift=-1.5cm]
      {\textbf{Perform $\bm{S_N}$ calculation\\in $\bm{V_1}$ for $\bm{\phi^{tr,(i)}_g}$ \& $\bm{J^{tr,(i)}_g}$}};
    \node (4) [process, below of=3, yshift=-1.5cm]
      {\textbf{Evaluate transport corrections, $\bm{D^{s,(i)}_g}$ or $\bm{\vec{D}^{(i)}_g}$}};
    \node (5) [process, below of=4, yshift = -1.5cm]
      {\textbf{Perform neutron diffusion\\calculation in $\bm{V_0}$ with transport\\corrections in $\bm{V_1}$}};
    \node (6) [decision, below of=5, yshift = -1.8cm]
      {\textbf{Convergence\\check}};
    \node (7) [object, below of=6, yshift = -2cm]
      {\textbf{Final $\bm{\phi}$ and $\bm{k}$\\solutions obtained}};
    \draw [arrow] (1) -- node[anchor=east] {$\bm{\phi^{(0)}_g, J^{(0)}_g}$} (2);
    \draw [arrow] (2) -- node[anchor=east] {$\bm{\Psi^-_g}$ \textbf{along} $\bm{\partial V_1}$} (3);
    \draw [arrow] (3) -- node[anchor=east] {$\bm{\phi^{tr,(i)}_g, J^{tr,(i)}_g}$} (4);
    \draw [arrow] (4) -- node[anchor=east] {$\bm{D^{s,(i)}_g}$ or $\bm{\vec{D}^{(i)}_g}$} (5);
    \draw [arrow] (5) -- node[anchor=east] {$\bm{\phi^{(i)}_g, J^{(i)}_g}$} (6);
    \draw [arrow] (6) -- node[anchor=east] {\textbf{Yes}} (7);
    \draw [arrow] (6) -- ([shift={(2cm,0cm)}]6.east)-- node[anchor=west] {\textbf{No}} ([shift={(0.5cm,0cm)}]2.east)--(2);
  \end{tikzpicture}
  \caption{Algorithm flowchart for the hybrid $S_N$-diffusion method. $(i)$ denotes the iteration
  index.}
  \label{fig:algorithm}
\end{figure}

\subsubsection{$S_N$ Subsolver Boundary Conditions \& Buffer Zone}

The main challenge lies in determining appropriate boundary conditions for the $S_N$ subproblem.
Given that we want to limit the coverage of $V_1$ to the control rod region and its
immediate vicinity, $V_1$ should be sufficiently smaller than $V_0$, but large enough to capture
anisotropies in the flux due to the control rod. As a consequence, the boundaries $\partial V_1$ 
lie well within $V_0$. Crucially, there is currently no feasible method of generating
accurate boundary fluxes for an $S_N$ solver from a neutron diffusion flux solution. In 1-D, the
standard one-group $S_N$ method requires N/2
incoming boundary flux parameters per boundary mesh point for the N/2 neutron angular fluxes
flowing into $V_1$. The neutron diffusion method can produce at most one independent
incoming flux parameter per mesh point; this parameter is the neutron forward/backward current in
the $P_1$ approximation defined as:
%
\begin{align}
  J_{g,\pm} &= \frac{\phi_g}{4} \mp \frac{D_g}{2}\frac{d\phi_g}{dx} \label{eq:p1-j}
  \shortintertext{where}
  J_{g,\pm} &= \mbox{ neutron forward/backward current of group }g. \nonumber
\end{align}
%
Without additional information, the next best estimate is a uniformly
isotropic transmission of angular flux, i.e., all N/2 incoming angular fluxes $\Psi$ are equal in
magnitude. This is a good assumption far from the control rod, but will not produce accurate
fluxes if the boundary is too close to the control rod. This type of boundary condition is similar
to the white boundary condition, which describes the uniformly
isotropic reflection of particles at a boundary, except this case involves transmission rather than
reflection. Isotropic flux transmission at $x$ can be expressed mathematically for forward angular
fluxes as:
%
\begin{align}
  \sum^N_{n=N/2+1}w_n\mu_n\Psi(x,\mu_n) =& J_{+}(x) && (\mu_n>0) \nonumber \\
  \Psi(x,\mu_n)\sum^N_{n=N/2+1}w_n\mu_n =& J_{+}(x) && (\because \mbox{isotropic transmission})
  \nonumber \\
  \Psi(x,\mu_n) =& J_{+}(x)\Bigg/\sum^N_{n=N/2+1}w_n\mu_n
\end{align}
%
and backward angular fluxes as:
%
\begin{align}
  \sum^{N/2}_{n=1}w_n\mu_n\Psi(x,\mu_n) =& J_{-}(x) && (\mu_n<0) \nonumber \\
  \Psi(x,\mu_n)\sum^{N/2}_{n=1}w_n\mu_n =& J_{-}(x) && (\because \mbox{isotropic transmission})
  \nonumber \\
  \Psi(x,\mu_n) =& J_{-}(x)\Bigg/\sum^{N/2}_{n=1}w_n\mu_n \label{eq:sn-psi-j}
\end{align}
%
These boundary conditions for the $S_N$ sub-solver will generally yield some deviations in the
$\phi$ distributions since neutron fluxes in realistic reactor systems are at least slightly
anisotropic in most of the domain. However, in my preliminary investigations, the influence of
boundary conditions on the ratio $J$ and $\frac{d\phi}{dx}$ does not extend far from the boundaries
in optically thick media due to the strong scattering (e.g., graphite or molten salt) or
absorption effects (e.g., control rod). In the bulk regions far from the boundaries, the
scattering and absorption effects have the most significant influence on the flux. I observed that
the \glspl{SVDC} generated with the $S_N$ sub-solver within $V_1$ were accurate everywhere except
near $\partial V_1$. Therefore, $V_1$ must be large enough to provide transport corrections
through the \glspl{SVDC} and accommodate inaccurate \glspl{SVDC} near
$\partial V_1$. The inaccurate \glspl{SVDC} should then be discarded in favor of using
$P_1$-based diffusion coefficients as in the rest of $V_0$ not covered by $V_0$. The
subsequent neutron diffusion calculation with this mix of \glspl{SVDC} and $P_1$-based diffusion
coefficients should provide more accurate $\phi$ and $k$ estimates than a conventional
neutron diffusion calculation.

\section{Summary} \label{sec:hybrid-summary}
%
%In this chapter, I presented the theoretical basis for a hybrid $S_N$-diffusion method, a novel
%method for improving the neutron diffusion method for neutronics simulations of reactor
%systems containing highly neutron-absorbing regions. The hybrid method relies on \glspl{SVDC}
%derived from reference flux solutions of $S_N$ calculations to provide transport
%corrections to the neutron diffusion method. The hybrid method is less computationally intensive
%than the standalone $S_N$ neutron transport method because 1) $S_N$ calculations are limited to
%correction regions which cover a fraction of the overall geometry, 2) and the \glspl{SVDC} are
%observed to converge faster than the neutron fluxes in $S_N$ calculations.
%
%I established a framework for the hybrid method for coupling an $S_N$ sub-solver and a neutron
%diffusion solver through outer iterations by taking advantage of the weak dependence of the $S_N$
%flux solution on the approximate boundary conditions determined from the neutron diffusion flux
%solution. This framework could be extended to improve multischeme methods, which
%divide a problem domain into subdomains where different neutronics methods are applied. For
%instance, in an $S_N$-diffusion multischeme method, the $S_N$ method is applied in subdomains with
%significant transport effects while other subdomains are treated with the neutron diffusion method
%to reduce the overall computational effort required \cite{wang_rattlesnake_2021}. 
%
%I also developed an algorithm for identifying and discarding inaccurate \gls{SVDC} values, which
%are expected to occur near the correction region boundary. I demonstrated the hybrid
%method on 1-D graphite-moderated test cases modeled after the \gls{MSRE}. The hybrid method showed
%good agreement with the reference Monte Carlo and $S_8$ methods after eliminating neutron energy
%group discretization errors in test cases that included highly neutron-absorbing regions. The
%hybrid method had better accuracy in more homogeneous systems largely due to the larger effective
%transport correction regions through \glspl{SVDC}. Further work
%is needed to extend and verify the hybrid method to 2-D and 3-D neutronics simulations and to
%create a robust solution for determining \glspl{SVDC} around neutron flux peaks and troughs. 
